\begin{resumo}

Um dos grandes problemas enfrentados pelos pesquisadores é o estudo das doenças complexas, pois elas são poligênicas e multifatoriais, fazendo com que diferentes estudos apresentem baixa replicabilidade. Esse problema em sido abordado por métodos que realizam integração de dados entre expressão gênica e dados de rede PPI (\textit{Protein Protein Interaction Network}). Dentre eles destaca-se o método NERI que obteve bons resultados de replicabilidade. O método NERI baseia-se nas hipóteses da \textit{Network Medicine} combinadas com métodos de importância relativa e obteve bons resultados de replicabilidade. A importância relativa é uma forma de inferir a importância dos nós da rede a partir de um conjunto de nós conhecidos como sementes. Entretanto, este método carece de uma análise de robustez que avalie o quanto seus resultados são dependentes dos genes sementes. Neste trabalho, analisamos a robustez do método NERI com relação aos genes sementes visando avaliar o impacto da remoção destes durante a análise. Utilizamos as técnicas de \textit{leave-one-out} e validação cruzada na qual removemos alguns nós sementes e comparamos cada resultado com o resultado original. Com isso, avaliamos a similaridade das listas de transcritos utilizando o método estatístico da correlação de postos de Spearman. Observamos que a correlação variou de 0,75 até 0,99 para o leave-one-out e de…. para a validação cruzada com 5 grupos de 6 genes cada. Portanto, o método é considerado robusto (ou não) e recomendamos que …

Palavras chaves: Network Medicine, Validação Cruzada, Leave-one-out, Robustez.
\end{resumo}