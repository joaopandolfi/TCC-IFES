%% abtex2-modelo-trabalho-academico.tex, v-1.9.5 laurocesar
%% Copyright 2012-2015 by abnTeX2 group at http://www.abntex.net.br/ 
%% 
%% This work may be distributed and/or modified under the
%% conditions of the LaTeX Project Public License, either version 1.3
%% of this license or (at your option) any later version.
%% The latest version of this license is in
%%   http://www.latex-project.org/lppl.txt
%% and version 1.3 or later is part of all distributions of LaTeX
%% version 2005/12/01 or later.
%%
%% This work has the LPPL maintenance status `maintained'.
%% 
%% The Current Maintainer of this work is the abnTeX2 team, led
%% by Lauro César Araujo. Further information are available on 
%% http://www.abntex.net.br/
%%
%% This work consists of the files abntex2-modelo-trabalho-academico.tex,
%% abntex2-modelo-include-comandos and abntex2-modelo-references.bib
%%

% ------------------------------------------------------------------------
% ------------------------------------------------------------------------
% abnTeX2: Modelo de Trabalho Academico (tese de doutorado, dissertacao de
% mestrado e trabalhos monograficos em geral) em conformidade com 
% ABNT NBR 14724:2011: Informacao e documentacao - Trabalhos academicos -
% Apresentacao
% ------------------------------------------------------------------------
% ------------------------------------------------------------------------

\documentclass[
	% -- opções da classe memoir --
	12pt,				% tamanho da fonte
	openright,			% capítulos começam em pág ímpar (insere página vazia caso preciso)
	oneside,			% para impressão em verso e anverso. Oposto a oneside
	a4paper,			% tamanho do papel. 
	% -- opções da classe abntex2 --
	%chapter=TITLE,		% títulos de capítulos convertidos em letras maiúsculas
	%section=TITLE,		% títulos de seções convertidos em letras maiúsculas
	%subsection=TITLE,	% títulos de subseções convertidos em letras maiúsculas
	%subsubsection=TITLE,% títulos de subsubseções convertidos em letras maiúsculas
	% -- opções do pacote babel --
	english,			% idioma adicional para hifenização
	french,				% idioma adicional para hifenização
	spanish,			% idioma adicional para hifenização
	brazil				% o último idioma é o principal do documento
	]{abntex2}

% ---
% Pacotes básicos 
% ---
\usepackage{lmodern}			% Usa a fonte Latin Modern			
\usepackage[T1]{fontenc}		% Selecao de codigos de fonte.
\usepackage[utf8]{inputenc}		% Codificacao do documento (conversão automática dos acentos)
\usepackage{lastpage}			% Usado pela Ficha catalográfica
\usepackage{indentfirst}		% Indenta o primeiro parágrafo de cada seção.
\usepackage{color,soul}				% Controle das cores
\usepackage{graphicx}			% Inclusão de gráficos
\usepackage{microtype} 			% para melhorias de justificação
% ---
\usepackage{amsmath}
\usepackage{afterpage}
\usepackage{url}
\usepackage{float}

\usepackage{pdfpages}
\usepackage[section]{placeins}
\usepackage{amsfonts}
\usepackage{multirow}
%---Python Code
\usepackage{listings}

\usepackage{caption}
\usepackage{subcaption}

\definecolor{mygreen}{rgb}{0,0.6,0}
\definecolor{mygray}{rgb}{0.5,0.5,0.5}
\definecolor{mymauve}{rgb}{0.58,0,0.82}

\lstset{ %
  backgroundcolor=\color{white},   % choose the background color
  basicstyle=\footnotesize,        % size of fonts used for the code
  breaklines=true,                 % automatic line breaking only at whitespace
  captionpos=b,                    % sets the caption-position to bottom
  commentstyle=\color{mygreen},    % comment style
  escapeinside={\%*}{*)},          % if you want to add LaTeX within your code
  keywordstyle=\color{blue},       % keyword style
  stringstyle=\color{mymauve}, 
  numbers=left,
  stepnumber=1,    
  firstnumber=1,
  numberfirstline=true% string literal style,
   frame=top,frame=bottom,
   captionpos=t,
   literate=
  {á}{{\'a}}1 {é}{{\'e}}1 {í}{{\'i}}1 {ó}{{\'o}}1 {ú}{{\'u}}1
  {Á}{{\'A}}1 {É}{{\'E}}1 {Í}{{\'I}}1 {Ó}{{\'O}}1 {Ú}{{\'U}}1
  {à}{{\`a}}1 {è}{{\`e}}1 {ì}{{\`i}}1 {ò}{{\`o}}1 {ù}{{\`u}}1
  {À}{{\`A}}1 {È}{{\'E}}1 {Ì}{{\`I}}1 {Ò}{{\`O}}1 {Ù}{{\`U}}1
  {ä}{{\"a}}1 {ë}{{\"e}}1 {ï}{{\"i}}1 {ö}{{\"o}}1 {ü}{{\"u}}1
  {Ä}{{\"A}}1 {Ë}{{\"E}}1 {Ï}{{\"I}}1 {Ö}{{\"O}}1 {Ü}{{\"U}}1
  {â}{{\^a}}1 {ê}{{\^e}}1 {î}{{\^i}}1 {ô}{{\^o}}1 {û}{{\^u}}1
  {Â}{{\^A}}1 {Ê}{{\^E}}1 {Î}{{\^I}}1 {Ô}{{\^O}}1 {Û}{{\^U}}1
  {œ}{{\oe}}1 {Œ}{{\OE}}1 {æ}{{\ae}}1 {Æ}{{\AE}}1 {ß}{{\ss}}1
  {ű}{{\H{u}}}1 {Ű}{{\H{U}}}1 {ő}{{\H{o}}}1 {Ő}{{\H{O}}}1
  {ç}{{\c c}}1 {Ç}{{\c C}}1 {ø}{{\o}}1 {å}{{\r a}}1 {Å}{{\r A}}1
  {€}{{\EUR}}1 {£}{{\pounds}}1
}



\usepackage{caption}

\floatstyle{plaintop}
\restylefloat{figure} 
\newfloat{grafico}{tbp}{lok}[chapter]
\floatname{grafico}{Gráfico}
\def\graficoautorefname{Gráfico}


% TABLE - FIRST ROW BOLD
\usepackage{array}
\newcolumntype{$}{>{\global\let\currentrowstyle\relax}}
\newcolumntype{^}{>{\currentrowstyle}}
\newcommand{\rowstyle}[1]{\gdef\currentrowstyle{#1}%
  #1\ignorespaces
}


\newcommand\ifes{Instituto Federal do Espírito Santo}
\newcommand*\erro[1]{\@latex@error{Defina \noexpand#1!}\@ehc}

\providecommand\imprimircurso{\erro\curso}
\newcommand*\curso[1]{\renewcommand{\imprimircurso}{#1}}

\providecommand\imprimirdepartamento{}
\newcommand*\departamento[1]{\renewcommand{\imprimirdepartamento}{#1}}
		
% ---
% Pacotes adicionais, usados apenas no âmbito do Modelo Canônico do abnteX2
% ---
\usepackage{lipsum}				% para geração de dummy text
\usepackage[T1]{fontenc}
\usepackage{inconsolata}

\usepackage{color}

\definecolor{pblue}{rgb}{0.13,0.13,1}
\definecolor{pgreen}{rgb}{0,0.5,0}
\definecolor{pred}{rgb}{0.9,0,0}
\definecolor{pgrey}{rgb}{0.46,0.45,0.48}

\usepackage{listings}
\lstset{language=Java,
  showspaces=false,
  showtabs=false,
  breaklines=true,
  showstringspaces=false,
  breakatwhitespace=true,
  commentstyle=\color{pgreen},
  keywordstyle=\color{pblue},
  stringstyle=\color{pred},
  basicstyle=\ttfamily,
  moredelim=[il][\textcolor{pgrey}]{$$},
  moredelim=[is][\textcolor{pgrey}]{\%\%}{\%\%}
}

\renewcommand{\lstlistingname}{Código}
% ---

% ---
% Pacotes de citações
% ---
\usepackage[brazilian,hyperpageref]{backref}	 % Paginas com as citações na bibl
\usepackage[alf]{abntex2cite}	% Citações padrão ABNT

% Put % before of what you want disabled

% Select what to do with todonotes: 
% \usepackage[colorinlistoftodos,disable]{todonotes} % notes not showed
  % notes showed

\usepackage[colorinlistoftodos,prependcaption,textsize=tiny]{todonotes}
\newcommand{\incerto}[1]{\todo[linecolor=red,backgroundcolor=red!25,bordercolor=red,#1]{#1}}
\newcommand{\alterar}[1]{\todo[linecolor=blue,backgroundcolor=blue!25,bordercolor=blue]{#1}}
\newcommand{\info}[1]{\todo[linecolor=OliveGreen,backgroundcolor=OliveGreen!25,bordercolor=OliveGreen]{#1}}


\usepackage[framemethod=tikz]{mdframed}

\newcommand{\melhorar}[2]{\todo{#1}\begin{mdframed}[hidealllines=true,backgroundcolor=blue!20]
#2
\end{mdframed}}

%Comando para inicar do lago esquerdo
\newcommand*\cleartoleftpage{%  
  \ifodd\value{page}\hbox{}\newpage\fi
  \clearpage
}



%

\usepackage{chngcntr}
\counterwithin{figure}{chapter}
\counterwithin{table}{chapter}


% --- 
% CONFIGURAÇÕES DE PACOTES
% --- 


%ajuste capa
\renewcommand{\imprimircapa}{
        \begin{capa}
          \center
          {\large\MakeUppercase\ifes}\par
          {\large\MakeUppercase\imprimirdepartamento}\par
          {\large\MakeUppercase\imprimircurso}\par
          \vfill
          {\large\MakeUppercase\imprimirautor}\par
          \vfill
          {\bfseries\large\MakeUppercase\imprimirtitulo}\par
          \vfill
         
          \vfill
          {\large\MakeUppercase\imprimirlocal}\par
          {\large\imprimirdata}\par
        \end{capa}
}


%\begin{comment}


\renewcommand{\imprimirfolhaderosto}{
        \begin{capa}
          \center          
          {\large\MakeUppercase\imprimirautor}\par
          \vfill
          {\bfseries\large\MakeUppercase\imprimirtitulo}\par
          \vspace{10 mm}
         \hspace{.45\textwidth}
         \begin{minipage}{.5\textwidth}
       \SingleSpacing
         \imprimirpreambulo 
         \\
         \\
         \imprimirorientadorRotulo
         
        {\large  } \large\imprimirorientador\par
       \end{minipage}
        \vfill 
          \vfill
          {\large\MakeUppercase\imprimirlocal}\par
          {\large\imprimirdata}\par
        \end{capa}
}
%\end{comment}






% ---
% Configurações do pacote backref
% Usado sem a opção hyperpageref de backref
\renewcommand{\backrefpagesname}{Citado na(s) página(s):~}
% Texto padrão antes do número das páginas
\renewcommand{\backref}{}
% Define os textos da citação
\renewcommand*{\backrefalt}[4]{
	\ifcase #1 %
		Nenhuma citação no texto.%
	\or
		Citado na página #2.%
	\else
		Citado #1 vezes nas páginas #2.%
	\fi}%
% % ---

% ---
% Informações de dados para CAPA e FOLHA DE ROSTO
% ---
%\titulo{Aplicação de Técnicas de Reconhecimento de Texturas Coloridas no Processo de Classificação de Espécies de Tartarugas Marinhas}
\titulo{ANÁLISE DE ROBUSTEZ DO MÉTODO DE INTEGRAÇÃO DE DADOS NERI}
\autor{João Carlos Pandolfi Santana}
\local{Serra}
\data{2017}
\orientador[Orientador:]{Prof. Dr. Sérgio Nery Simões}
%\coorientador{Equipe \abnTeX}
\instituicao{%
  Instituto Federal do Espírito Santo
 }
 \curso{Bacharelado em Sistemas de Informação}
\tipotrabalho{Monografia (Graduação)}
% O preambulo deve conter o tipo do trabalho, o objetivo, 
% o nome da instituição e a área de concentração 
\preambulo{ Trabalho de Conclusão de Curso apresentado à Coordenadoria do Curso de Bacharelado em Sistemas de Informação do Instituto Federal do Espírito Santo, como requisito parcial para obtenção do título de Bacharel em Sistemas de Informação.}
% ---


% ---
% Configurações de aparência do PDF final

% alterando o aspecto da cor azul
\definecolor{blue}{RGB}{41,5,195}

% informações do PDF
\makeatletter
\hypersetup{
     	%pagebackref=true,
		pdftitle={\@title}, 
		pdfauthor={\@author},
    	pdfsubject={\imprimirpreambulo},
	    pdfcreator={LaTeX with abnTeX2},
		pdfkeywords={abnt}{latex}{abntex}{abntex2}{trabalho acadêmico}, 
		colorlinks=true,       		% false: boxed links; true: colored links
    	linkcolor=blue,          	% color of internal links
    	citecolor=blue,        		% color of links to bibliography
    	filecolor=magenta,      		% color of file links
		urlcolor=blue,
		bookmarksdepth=4
}
\makeatother
% --- 

% --- 
% Espaçamentos entre linhas e parágrafos 
% --- 

% O tamanho do parágrafo é dado por:
\setlength{\parindent}{1.3cm}

% Controle do espaçamento entre um parágrafo e outro:
\setlength{\parskip}{0.2cm}  % tente também \onelineskip

% ---
% compila o indice
% ---
\makeindex
% ---

% ----
% Início do documento
% ----
\begin{document}

% Seleciona o idioma do documento (conforme pacotes do babel)
%\selectlanguage{english}
\selectlanguage{brazil}

% Retira espaço extra obsoleto entre as frases.
\frenchspacing 

% ----------------------------------------------------------
% ELEMENTOS PRÉ-TEXTUAIS
% ----------------------------------------------------------
% \pretextual

% ---
% Capa
% ---
\imprimircapa
% ---

% ---
% Folha de rosto
% (o * indica que haverá a ficha bibliográfica)
% ---
\imprimirfolhaderosto
% ---
\clearpage
% ---
% Inserir a ficha bibliografica
% ---

% Isto é um exemplo de Ficha Catalográfica, ou ``Dados internacionais de
% catalogação-na-publicação''. Você pode utilizar este modelo como referência. 
% Porém, provavelmente a biblioteca da sua universidade lhe fornecerá um PDF
% com a ficha catalográfica definitiva após a defesa do trabalho. Quando estiver
% com o documento, salve-o como PDF no diretório do seu projeto e substitua todo
% o conteúdo de implementação deste arquivo pelo comando abaixo:
%
% \begin{fichacatalografica}
%     \includepdf{fig_ficha_catalografica.pdf}
% \end{fichacatalografica}


\begin{fichacatalografica}
	\sffamily
	\vspace*{\fill}					% Posição vertical
	\begin{center}					% Minipage Centralizado
	\fbox{\begin{minipage}[c][8cm]{14cm}		% Largura
	\small
	P149e \imprimirautor
	%Sobrenome, Nome do autor
	
	\hspace{0.5cm} \imprimirtitulo  / \imprimirautor. --
	\imprimirlocal, \imprimirdata-
	
	\hspace{0.5cm} \pageref{LastPage} p. : il. (algumas color.) ; 30 cm.\\
	
	\hspace{0.5cm} \imprimirorientadorRotulo~\imprimirorientador
	
	\hspace{0.5cm}
	\parbox[t]{\textwidth}{\imprimirtipotrabalho~--~\imprimirinstituicao,\\
    Coordenadoria de Informática, Curso Bacharelado em Sistemas de Informação, \imprimirdata.}\\
	
	\hspace{0.5cm}
		1. 
		2. 
		3. 
		I. 
		II. Instituto Federal do Espírito Santo.
		III. Título.			
        
	\end{minipage}}
	\end{center}
\end{fichacatalografica}
% ---


% ---
% Inserir errata
% ---
\begin{comment}

\begin{errata}
Elemento opcional da \citeonline[4.2.1.2]{NBR14724:2011}. Exemplo:

\vspace{\onelineskip}

FERRIGNO, C. R. A. \textbf{Tratamento de neoplasias ósseas apendiculares com
reimplantação de enxerto ósseo autólogo autoclavado associado ao plasma
rico em plaquetas}: estudo crítico na cirurgia de preservação de membro em
cães. 2011. 128 f. Tese (Livre-Docência) - Faculdade de Medicina Veterinária e
Zootecnia, Universidade de São Paulo, São Paulo, 2011.

\begin{table}[htb]
\center
\footnotesize
\begin{tabular}{|p{1.4cm}|p{1cm}|p{3cm}|p{3cm}|}
  \hline
   \textbf{Folha} & \textbf{Linha}  & \textbf{Onde se lê}  & \textbf{Leia-se}  \\
    \hline
    1 & 10 & auto-conclavo & autoconclavo\\
   \hline
\end{tabular}
\end{table}

\end{errata}
\end{comment}
% ---

% ---
% Inserir folha de aprovação
% ---

% Isto é um exemplo de Folha de aprovação, elemento obrigatório da NBR
% 14724/2011 (seção 4.2.1.3). Você pode utilizar este modelo até a aprovação
% do trabalho. Após isso, substitua todo o conteúdo deste arquivo por uma
% imagem da página assinada pela banca com o comando abaixo:
%
 \includepdf{aprovacao.pdf}
%

\includepdf{declaracao.pdf}

\begin{comment}
\cleardoublepage
\begin{folhadeaprovacao}




\begin{center}

    {\large\MakeUppercase\imprimirautor}

    \vspace*{\fill}\vspace*{\fill}
    \begin{center}
      {\bfseries\large\MakeUppercase\imprimirtitulo}\par
    \end{center}
    \vspace*{\fill}

    
      \hspace{.45\textwidth}
      \begin{minipage}{.5\textwidth}
       \SingleSpacing
         \imprimirpreambulo
       \end{minipage}%
       \vspace*{\fill}
    
 \begin{center}
  
  Aprovado em 05 de Julho de 2016 \\
  \end{center}  
   
   \begin{center}
  
  \bfseries\large\MakeUppercase{Comissão Examinadora}
  \end{center}
    \vspace*{\fill}

    \setlength{\ABNTEXsignwidth}{10cm}
   
   \assinatura{\textbf{\imprimirorientador} \\ Instituto Federal do Espírito Santo \\ Orientador} 
   \assinatura{\textbf{a} \\ Instituto Federal do Espírito Santo}
   \assinatura{\textbf{a} \\ Instituto Federal do Espírito Santo}

   \begin{center}
    \vspace*{0.5cm}
    {\large\imprimirlocal}
    \par
    {\large\imprimirdata}
    \vspace*{1cm}
  \end{center}
\end{center}  
\end{folhadeaprovacao}

% ---

%Declaração do Autor
\cleardoublepage
\begin{folhadeaprovacao}




  \begin{center}
  
  \bfseries\large\MakeUppercase{Declaração do Autor}
  \end{center} 
  
  Declaro, para fins de pesquisa acadêmica, didática e técnico-científica, que o presente Trabalho de Conclusão de Curso pode ser parcial ou totalmente utilizado desde que se faça referência à fonte e aos autores.

    \vspace{10 cm}
   
   \assinatura{\textbf{\imprimirautor} }    
      
   \begin{center}    
    {\imprimirlocal}, X de Julho de {\imprimirdata}
    \vspace*{1cm}
  \end{center}

\end{folhadeaprovacao}
\end{comment}}
%----

% ---
% Dedicatória
% ---
\cleardoublepage
\begin{dedicatoria}
   \vspace*{\fill}
   \centering
   \noindent
   \textit{ Aos meus pais }  \\
   \textit{ Aos xxx } \vspace*{\fill}
\end{dedicatoria}
% ---

% ---
% Agradecimentos
% ---


\cleardoublepage
\begin{agradecimentos}
Agradecimentos


\end{agradecimentos}

% ---

% ---
% Epígrafe
% ---
\begin{epigrafe}
    \vspace*{\fill}
	\begin{flushright}
		\textit{Texto motivador} \\ \\ Winston Churchill
	\end{flushright}
\end{epigrafe}


% ---

% ---
% RESUMOS
% ---

% resumo em português
\setlength{\absparsep}{18pt} % ajusta o espaçamento dos parágrafos do resumo
\begin{resumo}

Um dos grandes problemas enfrentados pelos pesquisadores é o estudo das doenças complexas, pois elas são poligênicas e multifatoriais, fazendo com que diferentes estudos apresentem baixa replicabilidade. Esse problema em sido abordado por métodos que realizam integração de dados entre expressão gênica e dados de rede PPI (\textit{Protein Protein Interaction Network}). Dentre eles destaca-se o método NERI que obteve bons resultados de replicabilidade. O método NERI baseia-se nas hipóteses da \textit{Network Medicine} combinadas com métodos de importância relativa e obteve bons resultados de replicabilidade. A importância relativa é uma forma de inferir a importância dos nós da rede a partir de um conjunto de nós conhecidos como sementes. Entretanto, este método carece de uma análise de robustez que avalie o quanto seus resultados são dependentes dos genes sementes. Neste trabalho, analisamos a robustez do método NERI com relação aos genes sementes visando avaliar o impacto da remoção destes durante a análise. Utilizamos as técnicas de \textit{leave-one-out} e validação cruzada na qual removemos alguns nós sementes e comparamos cada resultado com o resultado original. Com isso, avaliamos a similaridade das listas de transcritos utilizando o método estatístico da correlação de postos de Spearman. Observamos que a correlação variou de 0,75 até 0,99 para o leave-one-out e de…. para a validação cruzada com 5 grupos de 6 genes cada. Portanto, o método é considerado robusto (ou não) e recomendamos que …

Palavras chaves: Network Medicine, Validação Cruzada, Leave-one-out, Robustez.
\end{resumo}
% resumo em inglês
\captionsenglish



\begin{resumo}
    
Traduzir o resumo
\end{resumo}


    

\captionsbrazil
% ---

% ---
% inserir lista de ilustrações
% ---
\pdfbookmark[0]{\listfigurename}{lof}
\listoffigures*
\cleardoublepage
% ---


% ---
% inserir lista de Grafico
% ---
%\newlistof{listofgraficos}{grafico}{Lista de Gráficos}
%\renewcommand{\listgraficoname}{Gráfico}
%\renewcommand{\cftlistofgraficospresnum}{AAA~}


\begin{comment}

\end{}%\pdfbookmark[0]{\listgraficoname}{log}
\newcommand{\cftgraficopresnum}{AAA}
%\listofgraficos
\listof{grafico}{Lista de Gráficos}

\cleardoublepage
\end{comment}
% ---

% ---
% inserir lista de tabelas
% ---
\pdfbookmark[0]{\listtablename}{lot}
\listoftables*
\cleardoublepage
% ---

% ---
% inserir lista de abreviaturas e siglas
% ---
\begin{comment}


\begin{siglas}
  \item[ABNT] Associação Brasileira de Normas Técnicas
  \item[abnTeX] ABsurdas Normas para TeX
\end{siglas}
\end{comment}
% ---

% ---
% inserir lista de símbolos
% ---
\begin{comment}
\begin{simbolos}
  \item[$ \Gamma $] Letra grega Gama
  \item[$ \Lambda $] Lambda
  \item[$ \zeta $] Letra grega minúscula zeta
  \item[$ \in $] Pertence
\end{simbolos}
\end{comment}
% ---

% ---
% inserir o sumario
% ---
\pdfbookmark[0]{\contentsname}{toc}
\tableofcontents*
\cleardoublepage
% ---



% ----------------------------------------------------------
% ELEMENTOS TEXTUAIS
% ----------------------------------------------------------
\textual

% ----------------------------------------------------------
% Introdução (exemplo de capítulo sem numeração, mas presente no Sumário)
% ----------------------------------------------------------
\captionsetup[figure]{justification=justified,singlelinecheck=false}

%tabulacao para a listagem
\newcommand{\itab}[1]{\hspace{0em}\rlap{#1}}
\newcommand{\tab}[1]{\hspace{.2\textwidth}\rlap{#1}}
%inicio do capitulo
\chapter[Introdução]{Introdução}

%\textit{} // Coloca em italico
%\cite{} //Cita autor
%\ref{} //Cita figura

Doenças complexas são poligênicas e multifatoriais, ou seja, além de serem causadas por mutações em mais de um gene, também são influenciadas por fatores ambientais. % \cite{davey-mith}. 
Alguns exemplos de doenças complexas são: esquizofrenia, transtorno do espectro autista, hipertensão, asma, Diabetes Melitus, doença de Parkinson e esclerose múltipla. % \cite{Hunter-2005}.
Quanto aos fatores genéticos, devido ao fato destas doenças serem poligênicas, as mutações podem levar a uma propagação não natural de informação e sinais, de forma que afete outros genes e/ou mecanismos dependentes dos que sofreram determinada mutação.
%
Uma forma de estudar este tipo de doença, é analisar os transcritos gerados pela transcrição dos genes, de forma a buscar uma relação de co-expressão, tendo como objetivo encontrar genes que influenciam na doença em questão.
Para utilizar estes dados, é possível modelar em forma de rede, onde cada nó representa um gene, as arestas representam a co-expressão genica, e ao utilizar a topologia de redes com pesos, o fator de co-expressão torna-se então o peso, determinando assim o grau de relacionamento entre dois nós, com essa abordagem, é possível aplicar conceitos e propriedades de grafos no problema, devido ao fato de ele estar modelado em rede.
Outra forma de estudar as doenças poligênicas, é analisar as interações entre proteínas (\textsl{PPI – Protein-Protein Interaction}), onde também é aplicada a abordagem de redes para investigação da doença, no qual é chamada de hipótese da \textsl{Network Medicine} \cite{Barabasi2011}. Este modelo leva em conta o nível de interação entre as proteínas e quais foram os genes responsáveis por gerá-las, podendo assim ter um mapeamento gênico e proteico ao mesmo tempo.

Estas duas abordagens citadas englobam conceitos de redes complexas, onde têm-se a representação de dados e relações entre eles em forma de grafos, sejam eles com pesos ou não (em sua grande maioria são utilizados grafos direcionados e com peso), esta abordagem permite utilizar conceitos fundamentados sobre teoria de grafos e algoritmos consolidados para análise do problema, ganhando-se assim mais ferramentas para tratamento do modelo em questão.

De acordo com os conceitos apresentados, existem diversas abordagens para tratar doenças poligênicas, dentre elas, destaca-se o método NERI \cite{Simoes2015} que apresentou bons resultados de replicabilidade. Este é um método que baseia-se em importância relativa, ou seja, fundamenta-se em nós sementes para o seu funcionamento, onde estes são genes sabidamente reconhecidos como importantes. Em vista desta abordagem, este método carece de uma análise de robustez, o que significa analisar o quão dependente dos nós sementes o método é, de forma a encontrar um coeficiente de confiança, para assim gerar uma segurança na utilização da ferramenta, porém, para encontrar estes coeficientes é necessário observar o comportamento do método quando há retiradas de nós sementes da rede de entrada.

Ao analisar o código fonte do programa em questão, foi identificada a necessidade de reestruturação do mesmo, de forma que fique mais modularizado, facilitando a manutenção e adição de novas funcionalidades futuras. 
%
Como a ferramenta foi desenvolvida para ser utilizada por biólogos, foi identificado também, a necessidade de desenvolver uma interface gráfica para facilitar e disseminar o uso.
Atualmente a utilização é feita somente por linha de comando no terminal, o que requer um nível de conhecimento mínimo. A interface gráfica tem como objetivo viabilizar o uso do programa para biólogos não familiarizados em utilizar programas por linhas de comando em terminal, potencializando assim o alcance da ferramenta e incentivando o uso de métodos computacionais por biólogos.


%Exemplo de imagem

%\textbf{\begin{figure}[ht!]
%\centering
%\includegraphics[width=130mm]{Images/debit.jpg}
%\caption {A evolução das máquinas de cartão. \label{adas}}
%\flushleft{Fonte: Imagem de retirada e adaptada de \cite{evolu}.}
%\end{figure}}

%\begin{figure}[ht!]
%\centering
%\includegraphics[width=130mm]{Images/debit.jpg}
%\caption {A evolução das máquinas de cartão. \label{Evolucao}}
%\flushleft{Fonte: Imagem de retirada e adaptada de \cite{evolu}.}
%\end{figure}


%\section{OCR} // Criando sessao



\section{Objetivos}
\subsection{Objetivo Geral}
\begin{center}
  \begin{enumerate}
  \item {Analisar robustez do método NERI avaliando o impacto da retirada de alguns genes sementes.}

  \end{enumerate}
\end{center}

\subsection{Objetivos Específicos}
\begin{center}
  \begin{enumerate}
  \item {Refatorar o código para facilitar a manutenção, utilização e contribuição externa da comunidade de usuários e desenvolvedores.}
    \item {Implementar os algoritmos de validação \textsl{Leave-one-out Cross-validation} e \textsl{Repeated K-Fold Cross-validation} aplicados ao método NERI.}
    \item {Analisar a robustez do método NERI verificando sua dependência em relação aos nós sementes.}
    \item{Iniciar o projeto da implementação de uma interface gráfica para facilitar o uso da ferramenta, visando torná-la mais intuitiva e amigável aos pesquisadores da área biológica.}
  \end{enumerate}
\end{center}


\section{Organização do trabalho}

No capítulo 2, apresentamos alguns conceitos relacionados a grafos e como estes podem ser utilizados para representar redes biológicas. Também são apresentados conceitos relativos as hipóteses da \textsl{Network Medicine}, da redes PPI, de importância relativa através de sementes, e como tais conceitos são utilizados pelo método de integração de dados NERI.
No capítulo 3 apresentamos a metodologia que utilizamos para avaliar a robustez do método através da remoção de um único e de vários genes sementes.
No capítulo 4 apresentamos os resultados das experimentos e comparamos com relação aos primeiros genes priorizados no experimento original.
No capítulo 5, fazemos algumas considerações finais em relação à robustez do NERI e sugerimos alguns trabalhos futuros.

% ----------------------------------------------------------
% PARTE
% ----------------------------------------------------------
%\part{Preparação da pesquisa}
% ----------------------------------------------------------

% ----------------------------------------------------------
% PARTE
% ----------------------------------------------------------
%\part{Referenciais teóricos}
% ----------------------------------------------------------

% ---
% Capitulo de revisão de literatura
% ---
\chapter{Referencial Teórico}

\index{Referencial Teórico }
	 
% \section{Fundamentos Teóricos}
\par
    Para melhor compreensão do conteúdo apresentado neste trabalho, este capítulo tem como objetivo explicar os fundamentos conceituais apresentados, de forma que os conceitos fundamentais possam ser compreendidos para a evolução do conteúdo.
    
\section{Redes}

\subsection{Grafos}
Grafos são formas de estruturação de dados ligados, onde um dado elemento é denominado nó ou vértice, a sua relação com outro elemento é chamada de aresta.
Para exemplificação, tome como nós, duas cidades A e B, as estradas que ligam estas cidades representam as arestas, desta forma pode-se modelar as ligações entre as cidades como um grafo.

%Imagem

\subsection{Grafos com pesos}
São grafos que possuem um grau de importância (também chamado de peso) em cada aresta, este grau de importância tem significado apenas em nível de abstração, o que significa que não carrega nenhum significado predefinido, geralmente, exprime o quão relacionado um nó está com outro.

%Imagem

\subsection{Passeio}
É uma sequência específica de nós ligados, partindo de p e chegando em g. O comprimento do passeio é determinado pelo número de arestas.

\subsection{Caminho}
Assim como o passeio, é uma sequência específica de nós ligados, porém este não possui vértices repetidos, ou seja, não passa duas vezes pelo mesmo vértice. A distância do caminho é definida pela soma dos pesos em suas arestas, para grafos sem peso, a distância é definida pela quantidade de arestas presentes no caminho, implicitamente definindo o peso de cada aresta como 1 e executando a soma das mesmas. 

%Imagem

\subsection{Distância}
Em grafos com peso, é definida pela soma dos pesos das arestas em um determinado caminho.
Em grafos sem peso, é definida como a quantidade de arestas (aresta peso 1) em um determinado caminho.

%Imagem


\subsection{Hub}
Hub é um nó que possui muitas arestas, ou seja, um nó que se liga a muitos outros.
%Imagem


\subsection{Bridge}
O quão ponte o nó é em relação a dois Hubs, ou seja, se um nó conectar dois Hubs o mesmo é definido como bridge
%Imagem


\subsection{Menor caminho ou caminho mínimo}
Quando se trata de grafos o \textbf{caminho mínimo} é aquele que possui a menor distância entre dois nós (p e g). \cite{dikstra-floyd} [Dijkstra, 1959; Floyd, 1962]
%Imagem

\subsection{Redes complexas}
Explicação

%Imagem

% ==== FUNDAMENTOS BIOLOGICOS ====

\section{Fundamentos biológicos}

\subsection{Transcrição}
Conceito de transcrição e fundamento da genomica

%imagem

\subsection{Coexpressão de transcritos}
Explicação

\subsection{Doenças multifatoriais}
epresentam um fenótipo ou determinam a doença.
O que significa, a doença não é composta por um único pedaço de DNA sequenciado, mas sim por vários pedaços de locais separados. \cite{barabasi}

%Imagem

% ==== REDES BIOLÓGICAS ====

\section{Redes Biológicas}

\subsection{Representação de genes em rede}
Texto

\subsection{Relação de menor caminho}
Texto

\subsection{Co-expressão como peso}
Texto

\subsection{Conceito de genes e nós sementes}
Texto

% ==== ANÁLISE DE ROBUSTEZ ====

\section{Métodos de análise de robustez}

\subsection{Conceito de robustez}
Texto

\subsection{Importância da análise}
Texto

\subsection{Método de validação cruzada}
O método de validação cruzada, também chamado de estimativa de rotação, é uma técnica desenvolvida para avaliar a capacidade de generalização de um determinado modelo, em relação a um conjunto de dados. Este modelo analisa os resultados estatísticos de um agrupamento de dados definido, onde tem sido amplamente empregado em problemas no qual o objetivo da modelagem é  predição de dados, isto se dá por seu conceito principal consistir no particionamento dos dados de entrada em subconjuntos mutualmente exclusivos, onde uma parte destes serão revezados na alimentação do modelo a ser validado (grupo de treinamento), e a outra parte utilizados na validação.
A definição do método consiste na separação dos dados em subconjuntos, de forma que os elementos sejam diferentes em todos subconjutos, feito o agrupamento, estes são revezados na alimentação do modelo a ser validado, em cada passo faz-se uma análise estatística dos resultados obtidos.
<EQ MATEMATICA>
<referencia>



\subsection{Método Leave-one-out}

Leave-one-out é um modelo de validação cruzada, diferencia-se na formação de agrupamentos, neste modelo a quantidade de subconjutos é a quantidade de elementos presentes, desta forma cada subconjunto possui somente um elemento.
<DESENVOLVER>
<EQ MATEMATICA>
<REFERENCIA>

% ==== TRABALHOS CORRELATOS ====

\section{Trabalhos correlatos}

\subsection{Teses}

\subsubsection{Tese de doutorado Sérgio Nery Simões}
\cite{NERI}
Este trabalho é a referencia principal do meu projeto, pelo fato do método no qual analisei a robustez é apresentado e descrito nele.

Para entender doenças complexas, é necessário encontrar os genes que se relacionam com a mesma. Com a evolução em larga escala das tecnologias de sequenciamento do genoma e das medições de transcritos, assim como o conhecimento da interação presente entre proteína-proteína (PPI – Protein Protein Interaction), a pesquisa sobre doenças complexas vêm se tornando cada vez mais comum. Ao basear-se no paradigma do Network Medicine, as redes de interação proteína-proteína têm sido utilizadas para enfatizar os genes relacionados à doenças complexas levando em conta fatores topológicos. Porém este método é afetado diretamente pela literatura disponível, onde proteínas mais estudadas tendem a ter mais conexões na rede, fazendo com que diminua a qualidade dos resultados. Sendo assim, métodos que utilizam somente redes PPI não fornecem dados dinâmicos e específicos, dado que a topologia da rede não é exclusiva para uma única doença. No trabalho em questão, foi desenvolvido um método que prioriza genes e vias biológicas relacionados a uma dada doença complexa, através da abordagem de não somente redes PPI mas também transcritômica e genômica, sendo os dados integrados em uma única rede. Após a integração e construção da rede, aplicou-se o conceito da Network Medicine, encontrando caminhos mínimos que possuam maior co-expressão entre seus genes. Com este modelo foi desenvolvido dois escores de ranqueamento, onde um prioriza genes com maior alteração entre suas pontuações em cada condição, e o outro privilegia os genes com a maior soma destas pontuações. Desta forma a aplicação do método em a três estudos envolvendo de expressão da doença esquizofrenia, recuperou com sucesso genes diferencialmente co-expressos em duas condições diferentes, e juntamente evitou os erros de literatura presentes na rede PPI. Em paralelo, melhorou substancialmente a replicação de resultados pelo método aplicado aos três estudos, onde por métodos convencionais, não atingiam uma replicabilidade satisfatória.


\subsection{Redes complexas}
	\subsubsection{Exploring complex networks}
	\cite{Strogatz}
	
	\subsubsection{Algorithms for Estimating Relative Importance in Networks}
	\cite{White-Scott}
	
	\subsubsection{Linked}
	\cite{linked-barabasi}
	
\subsection{Biologia}

	\subsubsection{DNA methylation: a form of epigenetic control of gene expression}
	\cite{Lim-Derek}
	
	\subsubsection{DNA methylation and its basic function}
	\cite{Moore-lisa}
	
\subsection{Redes Biológicas}
	\subsubsection{Using graph theory to analyze biological networks}
	\cite{Pavlopoulos}
	Este paper contém os conceitos fundamentais de redes biológicas e uso de grafos para sua análise.
<Descrever artigo>


	\subsubsection{An Integrative Systems Medicine Approach to Mapping Human Metabolic Diseases}
	\cite{barabasi-lazlo}
	
	\subsubsection{Exploring the human diseasome: The human disease network }
	\cite{goh-kwang}
	
	\subsubsection{Network Medicine}
	\cite{barabasi}
	
	Neste trabalho é definido o conceito de Network Medicine, este no qual baseia-se o método NERI.
<Descrever artigo>



% ---
\chapter[Metodologia]{Metodologia}


% RASCUNHO (PROJETO DO CAPÍTULO)
% METODOLOGIA
% 3.1 MATERIAIS
%     -- (PPI[HPRD, MINT INTACT], EXP[KATO], SEMENTES[LINK+APENDICE])
% ==> REVISAR QUAL A IDEIA DO TCC, OU SEJA, O QUE VOCÊ DESEJA DESCOBRIR?
% ==> SE O METODO É ROBUSTO, OU SEJA, O QUANTO OS RESULTADOS VARIAM APOS ALTERAR A ENTRADA (SEMENTES) 
% ==> COMO REALIZOU OS EXPERIMENTOS?c

% 3.2 Métodos utilizado para análise de robustez
%       -- MÉTODO DE REMOÇÃO DE UM ÚNICO GENE (METODO SIMILAR AO LEAVE ONE OUT)
%       -- MÉTODO DE REMOÇÃO DE VÁRIOS (SIMILAR AO CROSS VALIDATION) ==> 

% 3.2 VALIDAÇÃO (RODOU ALGUM EXPERIMENTO SEM REMOVER NINGUEM E COMPAROU COM OS RESULTADOS ORIGINAIS?)
% 



%
Para análise de robustez do \textsl{\textbf{Método NERI}} foram utilizados conceitos de validação baseados na alteração dos genes sementes. Para que seja feita a análise foram utilizados os resultados de priorização gênica gerada pela ferramenta. De forma a ser possível verificar e mapear os impactos causados devido a remoção ou falta de genes sementes no experimento. 

\section{Materiais}

%
Este trabalho utilizou a base de dados \textsl{\textbf{KATO}} que consiste em expressões gênicas de pessoas portadoras da doença \textsl{Esquizofrenia}. Estes dados podem ser encontrados \textcolor{red}{<AQUI>}. Esta base de dados em específico, foi selecionada devido ao fato de ter sido utilizada na tese de doutorado no qual este trabalho se baseia e se referencia. Assim sendo, os dados obtidos podem ser comparados com os encontrados e apresentados pelo autor, evitando o enviesamento do resultado por diferença de experimentação.

%
O \textsl{\textbf{Método NERI}} também recebe dados de rede de integração proteína proteína \textsl{(Protein Protein Interaction – PPI)}, onde os mesmos também foram mantidos os originais utilizados pelo autor, sendo formada pelas pelas bases de dados \textsl{\textbf{HPRD}}\footnote{HPRD: (http://www.hprd.org/)} (\textsl{Human Protein Reference Database}), \textsl{\textbf{MINT}}\footnote{MINT: (http://mint.bio.uniroma2.it/)} (\textsl{Molecular INTeraction database}) e \textsl{\textbf{IntAct}}\footnote{IntAct: (http://www.ebi.ac.uk/intact/)} (\textsl{IntAct molecular interaction database}).
%
Como entrada do sistema, também são definidos os \textsl{\textbf{Genes Sementes}}. Estes no qual, são os genes onde há certeza da sua relação com a doença analisada em questão, a base de dados original pode ser encontrada no apêndice~\ref{table_original_seeds}.

\section{Estratégia utilizada para análise de robustez}

%
Neste trabalho, foi escolhida a variação dos \textsl{\textbf{Genes Sementes}} como parâmetro de validação do \textsl{\textbf{Método NERI}}. Onde o objetivo é identificar o impacto gerado na rede de integração gênica e no resultado final (\textsl{Lista de priorização gênica}). Desta forma, podendo calcular a dependência e sensibilidade do método em relação a qualidade e quantidade de genes sementes.



\section{Escolha dos métodos de validação}

%
Os métodos de validação adotados foram selecionados pelas suas caraterísticas de estudo do problema em questão, não podendo deixar margem para enviesamento dos resultados e serem capazes de explorar comportamentos diferentes no experimento. Os modelos de validação escolhidos foram baseados no \textit{Leave one Out} e \textit{Cross Validation}.


\subsection{Remoção de um único gene semente}

Foi utilizado um modelo baseado no método validação \textsl{Leave one out}, onde consiste na remoção de um gene semente por vez. Desta forma, formando um \textsl{\textbf{conjunto diferente}} para cada \textsl{\textbf{gene semente}} removido, garantindo que sempre haja a mesma quantidade de genes em cada conjunto e que todos \textsl{\textbf{genes sementes}} tenham ficado de fora pelo menos uma vez. Assim sendo, o número final de conjuntos formados será a quantidade total de genes sementes.

%
Com este método, é possível descobrir se a falta de um único gene semente é responsável por alterar significantemente a priorização gênica gerada pelo \textsl{\textbf{Método NERI}}. Desta forma, podendo analisar se o método em questão é sensível a retiradas de \textsl{\textbf{genes sementes}}. 
O método de \textsl{\textbf{remoção de um único gene}} também permite a análise de importância relativa dos \textsl{\textbf{genes sementes}}, onde aquele que causar maior impacto no resultado final indica uma importância relativa maior em relação aos outros. 

%
Também é possível observar se existe relação entre as medidas de centralidade do gene representado na rede (neste trabalho, utilizaremos somente o grau do nó), com o impacto causado. Este é um ponto importante a ser analisado, pelo fato de poder estimar a importância relativa de um gene semente, observando o seu grau na rede de integração. Esta é uma medida que poderá impactar no resultado da priorização gênica resultante do \textsl{\textbf{Método NERI}}, por isso a importância da análise.


%Porém há outra análise importante que deve ser feita mas o Leave One Out não é capaz de prover, é se a quantidade de nós removidos influenciam diretamente no resultado. Para observar este aspecto, utilizamos o método Cross Validation, no qual, suas características se moldam mais a esta ótica de estudo.


\subsection{Remoção de vários genes sementes}

Foi utilizado um modelo baseado no \textsl{Cross Validation}, que consiste na remoção de mais de um gene semente por experimento.
%
Este método foi adotado pela sua característica principal, organização de conjuntos de genes sementes com tamanhos variados. Com esta característica chave, buscou-se estudar o comportamento da rede quando há a remoção de mais de um gene semente do conjunto original.

%
Com a formação de agrupamentos de genes sementes aleatórios e de tamanhos variados, poderá observar o comportamento do método analisado em situações variadas. Também será possível buscar o ponto de ruptura de proximidade ao resultado original, e assim estimar um grau de dependência e sensibilidade a uma quantidade ou arranjo de genes sementes presentes no experimento.

%
O fato dos agrupamentos possuírem arranjos de genes diferentes, abre-se a possibilidade de estudo sobre a eficácia de um ou mais genes semente sobre o resultado final, ou seja, se um arranjo específico promove melhores resultados que os demais, sendo os conjuntos de mesmo tamanho.

% 
Também é possível a observação dos conjuntos no qual foram removidos genes que causaram alto impacto em sua remoção na etapa anterior. Estes conjuntos informarão se o impacto na remoção de um único gene é acumulativo, ou seja, se houver a remoção de mais de um gene semente com alto impacto em sua remoção, se o resultado do experimento em questão será proporcional ao estudo da análise anterior.

\section{Definição dos genes sementes utilizados}

%
Em seu trabalho, os autores \cite{SIMOES2015} utilizaram um conjunto de 38 genes sementes obtidos de um estudo de associação de esquizofrenia, apresentados na Tabela~\ref{table_original_seeds} presente no Apêndice \ref{appendice_tables}.
%
%A Tabela~\ref{table_original_seeds} apresentada no capítulo 4 apresenta estes \textsl{\textbf{38}} genes.
%
No entanto, como o \textsl{\textbf{Método NERI}} utiliza a rede \textsl{PPI}, durante a integração dos genes sementes, somente \textsl{\textbf{30}} genes foram integrados a rede. Estes nos quais foram utilizados neste trabalho, em vista que os genes sementes que não integraram não afetam o resultado final, ficando de fora dos conjuntos de genes sementes gerados para validação.

\subsection{Aplicação do método similar ao  \textsl{Leave-one-out Cross-Validation}}

Para a validação utilizando o Método similar ao \textsl{Leave-one-out Cross-Validation}, em que neste trabalho chamaremos de \textsl{\textbf{Método de remoção de apenas um gene}}, a preparação do experimento consistiu em gerar conjuntos de genes sementes para o \textsl{\textbf{Método NERI}} de forma que cada conjunto tenha um gene a menos em relação ao experimento original. Foi criado para cada gene semente, um conjunto sem o mesmo, onde exista um conjunto para cada gene removido, ou seja, em uma amostra de \textsl{\textbf{30}} genes sementes, temos \textsl{\textbf{30}} experimentos possíveis.

A aplicação direta no sistema consiste em cada execução independente do programa. Onde cada experimento seja um conjunto formado da remoção de um gene semente da amostra original. Deve existir um experimento para cada gene semente removido, garantindo que cada gene semente tenha ficado de fora uma vez em relação a todos os experimentos.

%<Para preparação destas entradas, foi desenvolvido um script em Python 3.X para automatização do processo e para evitar falha humana.>

%<Script Python GERA LOO>

%
Para ilustração do experimento, segue o exemplo: Suponha que exista um conjunto de \{\textsl{\textbf{5}} genes sementes \{\textsl{\textbf{1,2,3,4,5}}\}, suponha que deseja-se gerar todos os possíveis conjuntos de genes sementes utilizando o método de \textsl{\textbf{remoção de apenas um gene}}. A Tabela~\ref{table_loo_example} representa os conjuntos resultantes. Seja \{\textsl{\textbf{$S_i$}} o i-ésimo subconjunto gerado.

%
% ======================= TABELA LOO EXEMPLO ==================
\begin{table}[]
\centering
\caption{Tabela representativa. Experimentos resultantes do método de remoção apenas um gene}
\label{table_loo_example}
\begin{tabular}{@{}lll@{}}
\toprule
\textbf{\textsl{Conjunto}} & \textbf{\textsl{Elementos}} \\ \midrule
\textsl{\textbf{$S_1$}} & 2,3,4,5 \\
\textsl{\textbf{$S_2$}} & 1,3,4,5 \\
\textsl{\textbf{$S_3$}} & 1,2,4,5 \\
\textsl{\textbf{$S_4$}} & 1,2,3,4 \\ \bottomrule
\end{tabular}
\flushleft{
Subconjuntos gerados pelo método de remoção de apenas um gene no conjunto de genes sementes \textbf{\{\textsl{1,2,3,4,5}\}}, sendo $S_i$ o i-ésimo subconjunto gerado.

Fonte: Tabela gerada pelo autor.}
\end{table}
%
% ======================= FIM TABELA LOO EXEMPLO ===============

\subsection{Aplicação do método similar ao \textsl{Repeated K-Fold Cross-Validation}}

Para a validação utilizando o Método similar ao \textsl{Repeated K-Fold Cross-Validation}, em que neste trabalho chamaremos de \textsl{\textbf{Método de remoção de mais de um gene}},
%
inicialmente, foi definido o tamanho dos grupos de genes sementes de entrada a serem removidos para cada bateria de execuções.
Levando em consideração a quantidade de genes de entrada total do experimento, \textsl{\textbf{30}} após a integração com a rede \textsl{PPI}, definiu-se que as remoções seriam feitas em relação a porcentagem da amostra original, sendo as porcentagens definidas \textbf{\textsl{10\%}} (3 genes), \textsl{\textbf{20\%}} (6 genes),\textsl{\textbf{30\%}} (9 genes) e \textsl{\textbf{40\%}} (12 genes), conforme pode ser observado na tabela \ref{table_gene_execucao}. Também representada na tabela, determinou-se a quantidade de agrupamentos de dados de entrada para cada porcentagem de remoção.

\begin{table}[]
\centering
\caption{Tabela representativa. Porcentagem de genes sementes removidos em relação aos \textsl{\textbf{30}} genes sementes originais e suas respectivas porcentagens}
\label{table_gene_execucao}
\begin{tabular}{@{}ccc@{}}
\toprule
\textbf{\textsl{Quantidade de}} & \textbf{\textsl{Porcentagem de}} & \textbf{Quantidade de} \\ 
\textbf{\textsl{genes removidos}} & \textbf{\textsl{remoção}} & \textbf{Experimentos} \\ \midrule
3 & 10\% & 50 \\
6 & 20\% & 50 \\
9 & 30\% & 50 \\
12& 40\% & 50 \\ \bottomrule
\end{tabular}
\flushleft{Fonte: Tabela gerada pelo autor.}
\end{table}

Para uma maior diversidade dos resultados, o método foi implementado de forma que cada grupo fosse diferente do outro.
%
Isso visa garantir que o conjunto de genes removidos não se repita dentro de cada etapa, para evitar executar o mesmo experimento mais de uma vez.
Assim, inicialmente são geradas todas as \textsl{\textbf{50}} combinações dos genes a serem removidos e em seguida são realizados os \textsl{\textbf{50}} experimentos.
%
O script que gera todas as combinações de N genes em M experimentos foi desenvolvido em \textsl{\textbf{Python 3.x}} e pode ser encontrado no Anexo~\ref{CVV_script}.


A Figura~\ref{cvv_explanation} ilustra o o processo de criação dos experimentos.
%
Para uma melhor compreensão, segue o exemplo abaixo.
Suponha que exista um conjunto de com 10 genes sementes: \{\textsl{\textbf{0,1,2,3,4,5,6,7,8,9}}\}, e suponha que deseja-se gerar 5 conjuntos de genes sementes com um percentual de remoção em \textsl{\textbf{20\%}}.
%
Assim, dado \textsl{\textbf{20\%}} de \textsl{\textbf{10}} elementos, temos \textsl{\textbf{fator de remoção = 2}} elementos.
Seja $S_i$ o i-ésimo subconjunto gerado cujo o respectivo conjunto de genes removidos é $R_i$
Portanto, um exemplo dos 5 subconjuntos está abaixo:

\begin{lstlisting}
    S1 = {0,1,2,3,4,5,6,7}
    S2 = {0,1,2,3,4,5,6,8}
    S3 = {0,1,2,3,4,5,6,9}
    S4 = {1,2,3,4,5,6,7,8}
    S5 = {2,3,4,5,6,7,8,9}
\end{lstlisting}

Sendo $R_i$ os conjuntos dos elementos removidos:

\begin{lstlisting}
    R1 = {8,9}
    R2 = {7,9}
    R3 = {7,8}
    R4 = {0,9}
    R5 = {0,1} 
\end{lstlisting}

%Imagem desenhada no caderninho
%Imagem
\begin{figure}[ht!]
\centering
\includegraphics[width=\textwidth]{Images/flows/cvv_explanation.png}
\caption {Fluxograma de funcionamento da remoção de vários genes sementes.
\label{cvv_explanation}}
\flushleft{Fonte: Produzido pelos autores.}
\end{figure}
%


Após os agrupamentos de dados de entrada serem preparados, o programa principal é executado individualmente para cada agrupamento. Totalizam-se 200 execuções individuais do programa que implementa o Método NERI.

\section{Execução dos experimentos}

%
% DETALHAR MAIS OS EXPERIMENTOS
% 
Nesta etapa, os experimentos encontram-se preparados para execução direta no programa principal que implementa o \textsl{\textbf{Método NERI}}.
%
Primeiramente, realizamos experimentos com a remoção de um único gene, para cada um dos \textsl{\textbf{30}} genes, o que totalizou \textsl{\textbf{30}} experimentos.
%
Em seguida, realizamos \textsl{\textbf{50}} experimentos para cada percentual de remoção (\textsl{\textbf{10\%}}, \textsl{\textbf{20\%}}, \textsl{\textbf{30\%}}, \textsl{\textbf{40\%}}), totalizando \textsl{\textbf{200}} experimentos nesta etapa.

%
Devido ao fato de o programa realizar cálculos demorados, houve a necessidade de automatização do processo de execução, onde foi desenvolvido um \textsl{Script} em \textsl{Shell} (linha de comando Linux), para efetuação do trabalho, podendo ser encontrado no apêndice~\ref{shell_run_all}.
Além disso, houve uma padronização nos diretórios utilizados pelo \textsl{\textbf{programa NERI}}, a fim de identificar os experimentos realizados, conforme pode ser observado na Tabela~\ref{}.
% DIRETORIO
% INPUT | DADOS DE ENTRADO DO USUARIO X Y E Z
% OUTPUT | RESULTADOS COM IDENTIFICADOR IGUAL AO DIRETORIO DE ENTRADA(+-)
% ...

%
Juntamente com as alterações estruturais, também foi desenvolvida uma \textsl{\textbf{interface gráfica}} e uma interface em \textsl{\textbf{linha de comando}} (\textsl{CLI}), para facilitar a utilização por pessoas que não possuem familiaridade com este modelo de execução de programas.
Essa interface está em fase de finalização, faltando a documentação e alguns ajustes, e assim que estiver finalizada será disponibilizada livremente na web.

\section{Validação}
%
Para garantir que não houvesse erro nos resultados devido a algum erro proveniente a configuração do ambiente de testes ou em relação as bases de dados utilizadas, foi executado o experimento original \textsl{\textbf{KATO}}. O resultado apresentado foi praticamente o mesmo, apresentando pequenos erros nas casas decimais \textsl{\textbf{$10^{-20}$}} provenientes por algum erro de arredondamento causado pela arquitetura do computador em questão, também pode ser levado em consideração atualizações das bibliotecas utilizadas pelo \textsl{\textbf{programa NERI}}. Estes erros foram muito baixos e não influenciaram no resultado final gerado pelo programa. O resultado comparativo dos experimentos bases podem ser observados na Tabela~\ref{}.

\section{Metologia de análise dos resultados}

% ==> REESCREVER
Após as execuções dos experimentos definidos, tivemos que definir as métricas para análise dos resultados.
O \textsl{\textbf{Método NERI}} realiza uma priorização gênica, produzindo como saída dois escores de ranqueamento ($\Delta'$ e $X$) para cada gene.
Assim, são produzidas duas listas (uma lista para o escore $X$ e outra para o escore $\Delta'$), cada uma contendo os genes que participaram dos caminhos mínimos selecionados. 

Assim, após a remoção dos genes sementes, realizamos uma comparação de ambas as listas resultantes com a lista original.
Essa comparação foi realizada tomando-se os primeiros genes N genes obtidos na lista do experimento e comparando-os com a respectiva lista original.
O valor N dos primeiros genes da lista foi variado em \{\textsl{\textbf{10}}, \textsl{\textbf{20}}, \textsl{\textbf{50}}, \textsl{\textbf{100}}\}, e para cada um foi feita a comparação da interseção.


% CHECAR AS PALAVRAS CORRELAÇÕES USADAS INDEVIDAMENTE
% remover o devido 
% ====== VERIFICAR SE VOU DEIXAR OU NAO O SPEARMAN
Para comparar duas listas ordenadas, geralmente utiliza-se a \textsl{\textbf{Correlação de Spearman}}.
Em nossos testes, inicialmente utilizamos essa correlação para compararmos as listas geradas pelos \textsl{\textbf{experimentos deste trabalho}} com a lista gerada pelo \textsl{\textbf{experimento original}}.
%
Observamos que, apesar de haver uma interseção razoável dos primeiros genes encontrados em ambas as listas,  a comparação utilizando a \textsl{\textbf{Correlação de Spearman}} apresentou valores próximos de zero, visto que esta medida penaliza listas que não estiverem na mesma ordem.
%
Assim, a \textsl{\textbf{Correlação de Spearman}} não mostrou-se um bom fator de comparação, pelo fato de a ordem dos genes em um grupo não ser a preocupação principal, mas sim se os genes foram escolhidos para estarem naquele agrupamento.


Muitas vez um biólogo pode estar mais preocupado com a \textsl{\textbf{replicabilidade}} de um experimento do que com a ordem dos genes priorizados.
Isto é, se os genes recuperados em um método são, em sua maioria, os mesmos recuperados em outro método.
Desta forma, algumas vezes pode ser mais importante comparar a interseção das listas do que a ordem das mesmas.
Assim, para a análise dos resultados, comparamos as interseções dos \textsl{\textbf{N}} primeiros genes de cada lista, variando \textsl{\textbf{N}} em \textsl{\textbf{\{10,20,50,100\}}}.


%====== FIM SPEARMAN

Foi utilizada a \textsl{\textbf{comparação por interseção}}, onde consiste em verificar a porcentagem de elementos presentes nas duas listas, de forma que o resultado indique o quão parecido os conjuntos são em relação a presença dos elementos iguais.
Sendo definido matematicamente por: 
$I = {{A \cup B} \over |A|}$

Este modelo de comparação beneficia as listas que possuírem mais elementos iguais, porém não penaliza as listas seccionadas que tiveram a ordem relativa dos elementos alterada. Este é um fator importante, pois o objetivo da análise é verificar a \textsl{\textbf{replicabilidade}} do experimento em condições diferenciadas, logo ligeiras variações de posicionamento dos elementos não são problemáticas, desde que os elementos das listas comparadas sejam iguais ou o possuírem uma grande interseção.


% ======================== EXEMPLOS TEX ========================
%Exemplo codigo no TEX
%\begin{lstlisting}[caption={Código fonte do método getBlockStatic da classe %Tools.},label=getBlockStatic,language=Java]
%public static List<Rect> getBlockStatic(Bitmap bmp, int nr, int nrTotal) {
%    Mat mat = bitmapToMat(bmp);
%    Size size = mat.size();
%    List<Rect> list = new ArrayList<>();
%    int left, top, rigth, bottom;
%    for (int i = 0; i < nrTotal; i++) {
%        top = 0;
%        if (i == 0) {
%            left = 1;
%        } else {
%            left = 1 + (mat.cols() / nrTotal * i);
%        }
%        bottom = mat.rows();
%        rigth = left + (mat.cols() / nrTotal);
%        list.add(new Rect(left, top, rigth - 1, bottom - 1));
%    }
%    List<Rect> result = new ArrayList<>();
%    for (int j = nr - 1; j < 4; j++) {
%        result.add(list.get(j));
%    }
%    list = null;
%    return result;
%}
%\end{lstlisting}

% ----------------------------------------------------------
% PARTE
% ----------------------------------------------------------
%\part{Resultados}
% ----------------------------------------------------------

\chapter[Experimentos, Resultados e Discussão]{Experimentos, Resultados e Discussão}

%%%%
%
% Gerador de tabelas: http://www.tablesgenerator.com/
%
% Elementos: http://www.tutorbrasil.com.br/forum/viewtopic.php?t=6163
%
% Sinônimos-> Assim sendo
% então, por conseguinte, deste jeito, desta maneira, dessarte, dessa forma, deste modo, desta forma, sendo assim, consequentemente, por isso, destarte, assim, portanto, logo, isto posto.
%
%
%%%%

Neste capítulo, avaliamos os impactos resultantes da remoção dos genes sementes nas listas de priorização resultante em comparação com a lista original. 
%
Ou seja, o quanto a lista de genes resultantes foram recuperados nos experimentos com remoção das sementes em relação a lista resultante original.

%Conforme mencionado no capítulo anterior, realizamos experimentos mantendo fixos a rede PPI e os dados de expressão, mas removendo progressivamente parte dos nós sementes do conjunto original (de forma similar as técnicas de avaliação de classificadores \textsl{leave-one-out} e validação cruzada), e comparando a interseção entre seus resultados com oresultado original. Variamos o percentual de genes sementes excluídos entre 10\% e 40\% e,em seguida, comparamos os primeiros elementos das listas resultantes com os primeiros da lista original. 


%
%===================== MEDIDAS DE CENTRALIDADE ===================
%

\section{Medidas de centralidade dos genes sementes}

Primeiramente, uma importante questão a ser verificada é se o impacto causado nos resultados devido à remoção de alguns genes sementes está correlacionado com alguma medida de centralidades dos respectivos genes na rede PPI.
Em outras palavras, tais medidas podem informar se o impacto da remoção dos genes sementes deve-se predominantemente a fatores topológicos.
%
A tabela \ref{centrality_measures} apresenta as medidas de centralidade que os genes sementes utilizados possuem na rede PPI.
Os genes estão apresentados ordenados pelo grau decrescente.
Observamos que os três primeiros genes: \textsl{TP53} (333), \textsl{AKT1} (138) e \textsl{DISC1} (91), possuem grau destacadamente maior que os demais, que possuem grau abaixo de 50.

% ============= Tabela com medidas de centralidade =========
\begin{table}[]
\centering
\caption{Medidas de centralidade dos genes sementes utilizados no experimento}
\label{centrality_measures}
\footnotesize
\begin{tabular}{@{}lrrrrrrr@{}}
\toprule
\textbf{\textsl{GENE}} & \textbf{Degree} &   \textbf{Betweenness} & \textbf{Closeness} & \textbf{Clustering} & \textbf{Brokering}  & \textbf{Bridgeness} \\ \midrule
\textbf{\textsl{TP53}}  & 333 & 1017443.366745 & 0.401408 & 0.027968 & 0.034839 &  135.635614 \\
\textbf{\textsl{AKT1}}  & 138 &  260150.217669 & 0.374562 & 0.044219 & 0.014196 &  206.018081 \\
\textbf{\textsl{DISC1}}  &  91 &  131642.895006 & 0.333142 & 0.016606 & 0.009632 &  130.539882 \\
\textbf{\textsl{FEZ1}}  &  43 &   39961.948306 & 0.315902 & 0.024363 & 0.004515 &  196.253855 \\
\textbf{\textsl{ERBB4}}  &  40 &   21172.397911 & 0.325737 & 0.144872 & 0.003682 &  188.374839 \\
\textbf{\textsl{GRIN2B}}  &  33 &   23614.446663 & 0.317337 & 0.090909 & 0.003229 &  362.909288 \\
\textbf{\textsl{APOE}}  &  29 &   19099.496744 & 0.318884 & 0.088670 & 0.002845 &  346.398103 \\
\textbf{\textsl{HP}}  &  21 &   22059.197543 & 0.324588 & 0.047619 & 0.002153 &  519.171834 \\
\textbf{\textsl{DRD2}}  &  17 &   15283.833355 & 0.301050 & 0.014706 & 0.001803 &  580.045729 \\
\textbf{\textsl{HTR2A}}  &  16 &    5847.333317 & 0.292464 & 0.008333 & 0.001708 &  212.879317 \\
\textbf{\textsl{IL1B}}  &  10 &    8530.925938 & 0.304214 & 0.066667 & 0.001005 & 1381.703074 \\
\textbf{\textsl{RGS4}}  &  10 &    2039.400569 & 0.292253 & 0.066667 & 0.001005 &  463.743523 \\
\textbf{\textsl{GAD1}}  &  10 &    7993.469172 & 0.317012 & 0.222222 & 0.000837 &  888.443883 \\
\textbf{\textsl{DRD1}}  &   8 &    9964.681533 & 0.282170 & 0.071429 & 0.000800 &  827.421927 \\
\textbf{\textsl{PPP3CC}}  &   8 &   10512.488952 & 0.289196 & 0.035714 & 0.000830 &  985.201738 \\
\textbf{\textsl{NRG1}}  &   7 &     349.958475 & 0.272072 & 0.238095 & 0.000574 &  126.388843 \\
\textbf{\textsl{COMT}}  &   5 &    3676.951004 & 0.273329 & 0.000000 & 0.000538 & 1028.239964 \\
\textbf{\textsl{SLC6A4}}  &   5 &     565.928439 & 0.283911 & 0.000000 & 0.000538 &  197.019715 \\
\textbf{\textsl{DRD4}}  &   5 &    2814.994398 & 0.298228 & 0.000000 & 0.000538 & 1209.842385 \\
\textbf{\textsl{PLXNA2}}  &   4 &    9316.180826 & 0.264610 & 0.000000 & 0.000431 & 1746.023442 \\
\textbf{\textsl{TPH1}}  &   4 &     574.162877 & 0.312481 & 0.333333 & 0.000287 & 7943.713227 \\
\textbf{\textsl{RELN}}  &   4 &      43.733810 & 0.240987 & 0.333333 & 0.000287 &   13.904311 \\
\textbf{\textsl{GRM3}}  &   4 &     248.812705 & 0.284468 & 0.000000 & 0.000431 &  597.497122 \\
\textbf{\textsl{GABRB2}}  &   4 &     555.585080 & 0.261145 & 0.000000 & 0.000431 &  288.445351 \\
\textbf{\textsl{DAO}}  &   3 &     768.669783 & 0.283496 & 0.000000 & 0.000323 &  676.604177 \\
\textbf{\textsl{OPCML}}  &   1 &       0.000000 & 0.253044 & 0.000000 & 0.000108 &    0.000000 \\
\textbf{\textsl{ZNF804A}}  &   1 &       0.000000 & 0.258773 & 0.000000 & 0.000108 &    0.000000 \\
\textbf{\textsl{MTHFR}}  &   1 &       0.000000 & 0.238457 & 0.000000 & 0.000108 &    0.000000 \\
\textbf{\textsl{RPGRIP1L}}  &   1 &       0.000000 & 0.270930 & 0.000000 & 0.000108 &    0.000000 \\
\textbf{\textsl{GRIK4}}  &   1 &       0.000000 & 0.229498 & 0.000000 & 0.000108 &    0.000000 \\ \bottomrule

\end{tabular}
\flushleft{Fonte: Tabela gerada pelo autor.}
\end{table}

% ========== FIM TABELA DE MEDIDAS DE CENTRALIDADE



\section{Remoção de um único gene semente}
%
Inicialmente, avaliamos o impacto da remoção de um único gene semente na lista resultante.
Esta avaliação foi realizada de forma similar ao método \textit{Leave One Out}.
A ideia deste experimento foi avaliar o impacto individual de cada gene no resultado final, com relação aos dois escores $X$ e $\Delta'$, obtidos durante a análise da rede diferencial no método \textsl{NERI}.
Em seguida, comparamos o impacto de tais resultados com as medidas de centralidade de redes para avaliar se há alguma correlação.
Desta forma, esta informação pode ser utilizada para lançar luz sobre o impacto das remoções de múltiplos genes sementes.

\subsection{Estudo dos gráficos em relação ao escore $\Delta'$}
%
% ============== S ===============
%
\subsubsection{Análise dos 10 primeiros elementos}
A figura \ref{fig_LOO_S_10} apresenta um gráfico comparativo dos experimentos utilizando o método de \textit{remoção de um único gene}, onde o eixo \textit{Horizontal} representa o gene removido em relação a amostra original, e o eixo \textit{Vertical}, por sua vez, representa a diferença percentual dos genes ranqueados em relação ao experimento original. Desta forma, comparando os 10 primeiros genes ranqueados relativos a remoção de cada gene apresentado, em relação aos 10 primeiros apresentados na amostra original, sendo o fator de ranqueamento o escore $\Delta'$.
%
%Imagem
\begin{figure}[ht!]
\centering
\includegraphics[width=\textwidth]{Images/analyses/fig_LOO_S_10.pdf}
\caption {Análise dos 10 primeiros elementos ordenados por $\Delta'$.
\label{fig_LOO_S_10}}
\flushleft{Fonte: Produzido pelos autores.}
\end{figure}
%

Podemos observar que os genes \textbf{\textit{TP53}} e \textbf{\textit{AKT1}}, apresentaram um maior impacto no resultado final, em vista que a diferença de percentual com o resultado original ficou, respectivamente, \textbf{\textit{40\%}} e \textbf{\textit{30\%}}. Isso implica que estes genes são de alta importância para o método atingir o resultado esperado.
%

Em contrapartida, os genes \textbf{\textit{IL1B, RELN, NRG1, GRIK4}} e \textbf{\textit{OPCML}} não apresentaram mudanças no resultado em relação ao escore analisada ($\Delta'$), assim como o gene \textbf{\textit{MTHFR}} e os outros que apresentaram \textbf{\textit{10\%}} de diferença dos genes selecionados, comparado ao resultado original do experimento. Com estes dados, podemos fazer a inferência de que os mesmos não apresentam uma importância significativa para o método em estudo em relação aos 10 primeiros selecionados utilizando o fator de ranqueamento o escore $\Delta'$.
%

Os genes \textbf{\textit{CHRNA7, DAOA, DTNBP1, MUTED, NPAS3, OFCC1, PRODH}} e \textbf{\textit{SLC18A1}} não foram integrados com a rede PPI.
Ou seja, durante a integração de dados, tais genes não possuíam um nó correspondente na rede PPI e, portanto, não foram utilizados.

Ao analisar os genes mencionados anteriormente \textbf{\textit{TP53}} e \textbf{\textit{AKT1}}, ambos possuem um alto grau na rede gerada pelo \textsl{Método NERI}.
E isto pode sugerir que a remoção de um gene com alto grau influencia diretamente no resultado.
%
Por outro lado, os demais genes deram valores bastante parecidos, oscilando entre 0.0 à 0.1.
Por exemplo, os genes ZNF804A, MTHFR e RPGRIP1L possuem grau 1, e no entanto, causaram impacto 0.1, semelhante ao gene terceiro gene DISC1 com grau 91.
Isso demonstra que a correlação entre o grau e o impacto observado na lista resultante final não é direta. 
%

%
\subsubsection{Análise dos 20 e 50 primeiros elementos}
%
%Imagem
\begin{figure}[ht!]
\includegraphics[width=1\textwidth]{Images/analyses/fig_LOO_S_20.pdf}
\includegraphics[width=1\textwidth]{Images/analyses/fig_LOO_S_50.pdf}
\caption {Análise dos 20 e 50 primeiros elementos ordenados por $\Delta'$.
\label{fig_LOO_S_20-50}}
\flushleft{Fonte: Produzido pelos autores.}
\end{figure}
%

A figura \ref{fig_LOO_S_20-50} apresenta dois gráficos de forma comparativa, de modo que o de cima representa os \textbf{\textit{20}} primeiros genes ranqueados resultantes em relação ao escore $\Delta'$ e o gráfico de baixo apresenta os \textbf{\texit{50}} primeiros. Sendo eixo \textit{Vertical} a similaridade com o resultado original e o eixo \textit{Horizontal} o gene removido no experimento em questão.
%

Esta comparação chama a atenção para a remoção do gene \textbf{\texit{MTHFR}}, que apresenta \textbf{\textit{0\%}} de impacto nos primeiros \textbf{\textit{20}} elementos, porém em contrapartida, apresenta \textbf{\textit{5\%}} de diferença entre os genes ranqueados em relação aos primeiros \textbf{\textit{50}} elementos. Isto se dá, devido ao fato da comparação de listas levar em conta os genes presentes no agrupamento ranqueado pelo experimento e pela amostra original. Consequentemente, o impacto de \textbf{\textit{0\%}} nos diz que em ambos agrupamentos (experimento e amostra original), possuem os mesmos genes ranqueados. Já no agrupamento ranqueado pelos \textbf{\textit{50}} primeiros elementos, não são todos os elementos similares, o que faz o fator de impacto subir de \textbf{\textit{0\%}} para \textbf{\textit{5\%}}.
%

Este mesmo efeito acontece na remoção do gene \textbf{\textit{RELN}}, variando de \textbf{\textit{0\%}} de diferença percentual, dos genes selecionados em relação ao experimento original, nos primeiros \textbf{\textit{20}} elementos para \textbf{\textit{5\%}} em relação aos primeiros \textbf{\textit{50}}.
%

Podemos observar também uma diminuição de experimentos com \textbf{\textit{10\%}} ou menos de impacto. Caindo de \textbf{\textit{36}} experimentos ao todo e \textbf{\textit{28}} válidos, para \textbf{\textit{31}} ao todo e \textbf{\textit{23}} válidos (Os experimentos não válidos para análise são os que não integraram com a base de dados \textbf{\textit{GWAS}}, totalizando \textbf{\textit{8}} \textit{genes/experimentos}).
%

O gene \textbf{\textit{TP53}} que causa o maior impacto na similaridade em sua remoção, variou de \textbf{\textit{25\%}} para \textbf{\textit{29\%}} nos respectivos agrupamentos \textbf{\textit{20}} e \textbf{\textit{50}} primeiros genes selecionados. Isso implica que ao analisar \textbf{\textit{30}} elementos a mais, houve um aumento do impacto de \textbf{\textit{4\%}} no pior caso. Sugerindo uma boa robustez do método em relação a remoção de um único gene semente.
%

Outro ponto importante de observação, é o gene \textsl{\textbf{AKT1}} que possui um alto grau na rede gerada pelo \textsl{\textbf{Método NERI}}, apresentar uma variação de impacto no resultado bem alta. O impacto apresentado vai de \textsl{\textbf{20\%}} para \textsl{\textbf{8\%}}, nos gráficos representantes dos \textsl{\textbf{20}} e \textsl{\textbf{50}} primeiros genes, respectivamente. Este fator aponta, mais um vez, que o grau do gene relativo a rede, não é um fator de impacto direto.

%
\subsubsection{Análise dos 100 e 200 primeiros elementos}
%
%Imagem
\begin{figure}[ht!]
\includegraphics[width=1\textwidth]{Images/analyses/fig_LOO_S_100.pdf}
\includegraphics[width=1\textwidth]{Images/analyses/fig_LOO_S_200.pdf}
\caption {Análise dos 100 e 200 primeiros elementos ordenados por $\Delta'$.
\label{fig_LOO_S_100-200}}
\flushleft{Fonte: Produzido pelos autores.}
\end{figure}
%
A imagem \ref{fig_LOO_S_100-200} também apresenta dois gráficos comparativos em relação a similaridade do ranqueamento gênico baseado na quantidade de elementos selecionados. O gráfico da esquerda, apresenta a diferença em percentual da similaridade dos \textbf{\textit{100}} primeiros genes e o da esquerda os primeiros \textbf{\textit{200}}.

%
Podemos observar em ambos os gráficos, que neste ponto de análise não houve remoção de gene que não causou impacto na similaridade dos resultados em relação a amostra original.

%
Um fato importante que ambos os gráficos demonstram, é a mediana dos impactos, onde apresentaram um valor abaixo de \textbf{\textit{10\%}}. Isso demonstra que o impacto gerado pela remoção de um único gene semente varia em torno de \textbf{\textit{10\%}}, o que é muito aceitável em vista da assertividade obtida.

%
Ao observarmos o gene \textbf{\textit{TP53}}, que durante todo o experimento foi o que apresentou maior impacto no resultado, podemos notar que o mesmo apresentou um aumento de \textbf{\textit{17\%}} do impacto para \textbf{\textit{24\%}} ao compararmos os ranqueamentos de \textbf{\textit{100}} e \textbf{\textit{200}} genes. Este valor pode ser considerado baixo, em vista que o numero de elementos da lista analisada foi dobrado e a diferença do impacto gerado foi de \textbf{\textit{7\%}}.

% 
Um comportamento que aparece em ambos os gráficos, que não havia aparecido com tanta nitidez nos anteriores, é a decrescência do impacto da esquerda para direita. Onde os experimentos estão ordenados com relação ao seu grau na rede interna gerada pelo \textsl{\textbf{Método NERI}}. Isso implica que mesmo não tendo uma relação direta com o impacto o grau do gene semente apresenta correlação ao analisar uma quantidade maior de genes priorizados do resultado final.

%
%
% ============== X ===============
%
%
\subsection{Estudo dos gráficos em relação ao escore \textit{X}}
%
\subsubsection{Análise dos 10 primeiros elementos}
%
%Imagem
\begin{figure}[ht!]
\centering
\includegraphics[width=\textwidth]{Images/analyses/fig_LOO_X_10.pdf}
\caption {Análise dos 10 primeiros elementos ordenados por \textit{X}.
\label{fig_LOO_X_10}}
\flushleft{Fonte: Produzido pelos autores.}
\end{figure}
%

%
A figura \ref{fig_LOO_X_10} apresenta um gráfico comparativo referente aos resultados dos experimentos com remoção de apenas \textsl{um gene semente} por vez. O eixo \textbf{Horizontal} representa cada experimento com seu respectivo \textbf{gene semente} removido do agrupamento original, estando estes ordenados  pelo grau que representa na rede gerada pelo \textsl{Método NERI}. Sendo esta organização, do maior para o menor no sentido esquerda para direita. O eixo \textsl{Vertical}, por sua vez, representa a \textsl{diferença percentual} dos genes ranqueados em relação ao experimento original, tendo como fator de ordenação o escore \textsl{\textbf{X}}. Por conseguinte, apresentando a comparação dos \textsl{10} primeiros genes ranqueados, sendo estes em relação a remoção do respectivo \textsl{gene semente} apresentado, com os 10 primeiros ranqueados pelo \textsl{experimento original}.
%

Em primeira análise, podemos perceber que o gene semente \textbf{\textsl{RPGRIP1L}} causou o maior impacto em sua remoção, apresentando o percentual de \textsl{\textbf{20\%}} de diferença dos \textsl{\textbf{genes ranqueados}} em relação a amostra original. Este impacto apresenta um comportamento de \textsl{\textbf{outlier}} em relação ao agrupamento de experimentos, em vista que a mediana dos impactos foi de \textsl{\textbf{0\%}}, ou seja, a maior parte dos experimentos em questão não apresentaram diferença entre os \textsl{genes ranqueados} com o \textsl{experimento original}.
%

Podemos observar também que apenas \textsl{\textbf{9}} genes apresentaram impacto em sua remoção. Sendo entre estes, a mediana do impacto \textsl{\textbf{10\%}}, o que é um bom indicador da robustez do \textsl{\textbf{Método NERI}}.
%

Um ponto importante para observação é a mudança do gene causador de maior impacto em relação aos \textsl{\textbf{fatores de ranqueamento}}. Quando o ranqueamento foi feito pelo escore \textsl{\textbf{$\Delta'$}} (analisado na seção anterior), o gene semente causador de maior impacto foi \textsl{\textbf{TP53}}, este no qual aprestou apenas \textsl{\textbf{10\%}} de impacto em relação ao escore de ranqueamento \textsl{\textbf{X}}. Isso indica que ambos os genes são impactantes no resultado final do experimento, porém, a abordagem adotada para ranqueamento gênico influencia diretamente na análise.
%


\subsubsection{Análise dos 20 e 50 primeiros elementos}
%
%Imagem
\begin{figure}[ht!]
\includegraphics[width=1\textwidth]{Images/analyses/fig_LOO_X_20.pdf}
\includegraphics[width=1\textwidth]{Images/analyses/fig_LOO_X_50.pdf}
\caption {Análise dos 20 e 50 primeiros elementos ordenados por \textit{X}.
\label{fig_LOO_X_20-50}}
\flushleft{Fonte: Produzido pelos autores.}
\end{figure}
%

%
A figura \ref{fig_LOO_X_20-50} apresenta dois gráficos comparativos, demonstrando o impacto causado no ranqueamento gênico pelo \textsl{\textbf{Método NERI}} ao remover determinados genes sementes. Cada gráfico representa o impacto relativo a quantidade dos genes priorizados analisados, de forma que o eixo \textsl{Horizontal} represente os genes removidos em cada experimento, estando ordenados da esquerda para direita levando em conta grau do gene semente. Assim sendo, o eixo \textsl{Vertical} indica o impacto causado no resultado final em relação a amostra original, este impacto se dá pela diferença percentual de genes presentes no agrupamento experimento e agrupamento original. Estão representados no gráfico da esquerda, o impacto causado em relação ao ranqueamento dos \textsl{\textbf{20}} primeiros genes, e os \textsl{\textbf{50}} primeiros no gráfico da direita.

%
Podemos observar em primeira observação, que a quantidade de genes que causaram impacto no ranqueamento gênico. Onde o impacto aumentou consideravelmente logo nos primeiros \textsl{\textbf{20}} genes analisados, sendo \textsl{\textbf{22}} experimentos impactantes. Diferente dos \textsl{\textbf{10}} primeiros, como pode ser visto no gráfico da figura \ref{fig_LOO_X_10}, onde apenas \textsl{\textbf{9}} experimentos apresentaram impacto. Este mesmo comportamento pode ser observado, ao compararmos o gráfico dos \textsl{\textbf{20}} primeiros genes, com o gráfico dos \textsl{\textbf{50}} primeiros. Este ultimo apresenta apenas \textsl{\textbf{1}} experimento que não causou impacto em sua remoção, sendo ele a remoção do gene \textsl{\textbf{TPH1}}.

%
Quando olhamos para o experimento com a remoção do gene \textsl{\textbf{TPH1}}, podemos notar um comportamento atípico. O mesmo apresentou um impacto no ranqueamento gênico de \textsl{\textbf{10\%}} nos primeiros \textsl{\textbf{10}} genes analisados. Porém, nos dois gráficos subsequentes, representando respectivamente \textsl{\textbf{20}} e \textsl{\textbf{50}} primeiros genes ranqueados, o mesmo não apresentou impacto. Ao observarmos este comportamento, podemos inferir que este gene não é de grande importância para o ranqueamento gênico feito pelo \textsl{\textbf{Método NERI}}.

%
O gráfico que representa os \textsl{\textbf{20}} primeiros genes ranqueados em relação a \textsl{\textbf{X}}, apresenta uma mediana de impacto de apenas \textsl{\textbf{5\%}}, valor este, considerado muito baixo, em vista que estes são os genes considerados mais importantes pelo \textsl{\textbf{Método NERI}}.

%
Uma outra ótica que podemos incutir aos experimentos é a observação em relação ao grau dos genes sementes removidos. Os gráficos demonstram que em relação ao escore \textsl{\textbf{X}}, o grau é um fator altamente impactante no resultado final, em vista que os genes com maior grau \textsl{\textbf{TP53}}, \textsl{\textbf{AKT1}} e \textsl{\textbf{DISC1}} não apresentaram em sua maioria os maires impactos no resultado final. Ficando esta métrica salva apenas para o \textsl{\textbf{TP53}} na análise dos \textsl{\textbf{50}} primeiros genes ranqueados, onde seu impacto é de \textsl{\textbf{22\%}} em relação a amostra original.

%


%
%
\subsubsection{Análise dos 100 e 200 primeiros elementos}
%
%Imagem
\begin{figure}[ht!]
\includegraphics[width=1\textwidth]{Images/analyses/fig_LOO_X_100.pdf}
\includegraphics[width=1\textwidth]{Images/analyses/fig_LOO_X_200.pdf}
\caption {Análise dos 100 e 200 primeiros elementos ordenados por \textit{X}.
\label{fig_LOO_X_100-200}}
\flushleft{Fonte: Produzido pelos autores.}
\end{figure}
%
%

A Figura~\ref{fig_LOO_S_100-200} apresenta dois gráficos comparativos em relação ao impacto representado pela remoção do \textsl{\textbf{gene semente respectivo}}. Estes gráficos podem ser lidos da mesma forma que os apresentados anteriormente nesta sessão, onde o da esquerda representa os primeiros \textsl{\textbf{100}} genes ranqueados e o da direta os \textsl{\textbf{200}} primeiros.
%

Um comportamento que podemos observar ao analisar os dois gráficos apresentados, é a diminuição do impacto geral causado pela remoção dos genes sementes encontrados nos primeiros \textsl{\textbf{200}} genes ranqueados. Esta diminuição no impacto se dá pelo aumento da lista de genes ranqueados, de forma que mais genes em comum estejam nas listas geradas pelos experimentos e na lista do experimento original. Este comportamento indica uma tendência de diminuição do impacto em relação ao tamanho da lista analisada, porém, o gene semente causador de maior impacto em sua remoção ainda se mantém, sendo \textsl{\textbf{TP53}} com \textsl{\textbf{12\%}} e \textsl{\textbf{10\%}} de impacto nos gráficos de \textsl{\textbf{100}} e \textsl{\textbf{200}} genes ranqueados respectivamente.

Também podemos notar um limiar de impacto abaixo de \textsl{\textbf{10\%}}, ou seja, a maioria dos experimentos causaram um impacto igual ou inferior a \textsl{\textbf{10\%}} no resultado final em relação a remoção do gene semente respectivo. Este comportamento é um bom indicativo para a robustez do método, em vista que o mesmo se mostra com baixa variação no resultado em relação a remoção de um único gene semente, quando o fator de ranqueamento é o escore $X$.

%
%
% ============== OBS ===============
%
%
\subsection{Observações}
%
Em relação ao método de validação de \textit{remoção de um único gene}, o \textit{método NERI} apresentou bons resultados de robustez. De forma que o maior impacto encontrado pela remoção de um único gene semente foi de \textit{\textbf{40\%}} em relação ao escore $\Delta'$. Porém  a mediana das correlações com o mesmo escore de ranqueamento, foi de \textit{\textbf{20\%}} em relação aos \textbf{\textit{10}} primeiros elementos. Estes valores melhoram ao observar o ranqueamento em relação ao escore $X$, apresentando o maior valor de impacto em \textsl{\textbf{20\%}} com o gene \textsl{\textbf{GRIP1L}}. Fato este que chama atenção por ser um gene semente diferente do maior causador de impacto em relação ao escore $\Delta'$, o gene semente \textsl{\textbf{TP53}}. A mediana de impacto nos primeiros \textsl{\textbf{10}} genes ranqueados pelo escore $X$ foi de \textsl{\textbf{0\%}}, ou seja, a remoção de mais da metade dos genes sementes individualmente não causou impacto no resultado final. Isso significa que os na maioria dos casos os genes ranqueados tanto nos experimentos quanto na amostra original foram os mesmos.

%
% Falar sobre a metrica X também
%

Também deve-se levar em conta que o maior impacto encontrado relação aos \textbf{\textit{200}} primeiros genes selecionados pelo escore $\Delta'$, foi de \textbf{\textit{24\%}} apresentando melhora em relação a análise dos \textbf{\texit{10}} primeiros. A medina apresentou uma queda significativa de \textbf{\texit{20\%}} para \textbf{\textit{5\%}}, o que indica uma convergência de genes selecionados da amostra original com os experimentos. Esse tipo de convergência é esperado com o aumento da quantidade de elementos ranqueados, pois a probabilidade de um gene ser selecionado aumenta proporcionalmente ao tamanho do agrupamento de seleção final. Porém, esta premissa não invalida a eficiência do método em questão, em vista que a quantidade de genes possíveis a serem selecionados é muito maior que a lista dos genes ranqueados.
%
Assim sendo, podemos concluir que o método é robusto em relação a retirada de um único gene semente. Porém, para determinar melhor a robustez do método em análise, há a necessidade de estudar os resultados dos outros modelos de validação empregadas neste trabalho.
%
%
% ======================== CROSS VALIDATION ========================
%
%
\section{Remoção de vários genes sementes}
%
Nesta etapa analisaremos os resultados do método de \textit{Remoção de vários gene sementes}, onde como principal meio de apresentação de dados será o estudo dos gráficos gerados e a discussão de suas interpretações.

\subsection{Esperado}
%
O esperado na utilização do método da \textit{Remoção de mais de um gene} é a capacidade de mapear o impacto causado no resultado final baseado na identificação da quantidade de genes sementes removidos em relação a amostra original. Desta forma observar o impacto causado, a medida que conjuntos de tamanhos diferentes são testados. Com este estudo, podemos aproximar um limiar de confiança no \textsl{\textbf{Método NERI}}. Para podermos ter uma análise mais precisa dos resultados, cruzaremos os dados encontrados com os dados gerados pela \textit{Remoção de um único gene}, de forma que consigamos entender melhor o comportamento dos resultados apresentados.
%
%
% ===== S =====
%
%
\subsection{Estudo dos gráficos em relação ao escore $\Delta'$}
%
\subsubsection{Análise dos 10 primeiros elementos}
%
%Imagem
\begin{figure}[ht!]
\centering
\includegraphics[width=\textwidth]{Images/analyses/fig_S_10_40.pdf}
\caption {Análise dos 10 primeiros elementos ordenados por $\Delta'$.
\label{fig_S_10_40}}
\flushleft{Fonte: Produzido pelos autores.}
\end{figure}
%

%
A Figura~\ref{fig_S_10_40} apresenta um gráfico comparativo dos experimentos utilizando o método \textit{Remoção de mais de um gene}, onde o eixo \textsl{Horizontal} representa a porcentagem de sementes excluídas em relação a amostra original, e o eixo \textsl{Vertical}, por sua vez, representa a porcentagem da interseção dos resultados dos \textsl{\textbf{10}} primeiros genes em relação aos \textsl{\textbf{10}} primeiros apresentados na amostra original, tendo como fator de ranqueamento ao escore \textsl{\textbf{$\Delta'$}}.
%

Conforme podemos observar, a medida que aumentamos o percentual de sementes excluídas, ocorre uma redução gradual na mediana do percentual de interseção de genes selecionados.
Conforme esperado, isso demonstra que a quantidade de genes sementes excluídas influencia diretamente no resultado do experimento.
%

%
Podemos observar também que, a mediana do experimento com a maior porcentagem de remoção apresenta o valor aproximado \textsl{\textbf{50\%}} de interseção, esta métrica é um bom indicativo da robustez do método ao informar que mesmo removendo \textsl{\textbf{40\%}} dos genes sementes, os genes ranqueados pelo método ainda se mantém acima de \textsl{\textbf{50\%}} iguais aos ranqueados pelo método com todos os genes sementes.

Por este gráfico representar somente os \textsl{\textbf{10}} primeiros genes selecionados, esperava-se um impacto no resultado final consideravelmente alto devido ao fato do mesmo apresentar poucos genes em relação ao tamanho da rede \textsl{\textbf{9.554}} Genes (nós). Porém, ao contrário do que imaginávamos, os genes selecionados foram muito próximos da amostra original. Porém a sua precisão varia consideravelmente de modo que a podemos observar que as diferenças entre os limites superiores e inferiores dos experimentos são altas. Isto se dá devido ao tamanho da lista de genes priorizados analisada.
%

Um ponto que chama bastante atenção ao analisar este gráfico, é o boxplot que representa \textbf{\textsl{40\%}} dos genes sementes removidos. O mesmo, apresenta uma variação entre o lime inferior e o limite superior de \textbf{\textsl{60\%}}, ou seja, a bateria de experimentos representados pelo gráfico possui experimentos de similaridade variada entre \textbf{\textsl{20\%}} a \textbf{\textsl{80\%}}. Fato este implica em uma não confiança nos dados representados por este, o comportamento apresentado é reforçado ao fazer uma análise dos \textbf{\textsl{outliers}}. Encontrando um experimento com \textbf{\textsl{90\%}} de similaridade e em contrapartida, um experimento com apenas \textbf{\textsl{10\%}} sendo o menor de todo o estudo em relação aos \textbf{\textsl{10}} primeiros genes removidos.

\subsubsection{Análise dos 20 e 50 primeiros elementos}
%Imagem
\begin{figure}[ht!]
\includegraphics[width=1\textwidth]{Images/analyses/fig_S_20_40.pdf}
\includegraphics[width=1\textwidth]{Images/analyses/fig_S_50_40.pdf}
\caption {Análise dos 20 e 50 primeiros elementos ordenados por $\Delta'$.
\label{fig_S_20-50_40}}
\flushleft{Fonte: Produzido pelos autores.}
\end{figure}
%

A Figura~\ref{fig_S_20-50_40} apresenta dois gráficos comparativos, sendo o da esquerda representando a correlação com os \textsl{\textbf{20}} primeiros genes selecionados e da direita com os primeiros \texit{\textbf{50}}. 
%

Podemos observar no primeiro gráfico a baixa variação da mediana, porém houve uma diminuição na \textit{amplitude interquartílica}. Isto demonstra uma possível convergência de resultados em relação aos dois gráficos. Este fator pode ser observado no \textit{boxplot} referente a \textbf{\textsl{20\%}} do gráfico da esquerda, onde este representa \textsl{\textbf{20}} primeiros genes selecionados. Neste caso, \textit{amplitude interquartílica} varia de \textit{68\%} a \textit{82\%}, totalizando \textit{14\%} de faixa de variação. No gráfico da esquerda, representando os \textbf{\textit{50}} primeiros genes selecionados. Neste caso, há uma variação na \textsl{amplitude interquartílica} de \textbf{\textsl{73\%}} a \textbf{\textsl{80\%}}, totalizando uma faixa de variação de \textbf{\textsl{7\%}}, valor este que apresenta-se metade do valor do gráfico anterior. Esta queda de amplitude remete ao comportamento de convergência, assim representando uma segurança nos resultados apresentados, partindo do princípio de que quanto menor a variação dos resultados, maior é a precisão da medição.
%

Podemos observar também alguns experimentos que ficaram fora do agrupamento, este comportamento é definido como \textit{outlier}. Para entender o porque destes experimentos terem sido apresentados unanimamente com menores resultados do que o corpo amostral, cruzamos os seus dados com os obtidos pela etapa de \textit{remoção de um único gene}. Com este cruzamento de dados, observamos se os genes removidos nestes experimentos contém um ou mais genes que possuem os maiores \textsl{\textsl{graus de impacto}} no resultado.
%

% VERIFICAR QUAIS FORAM OS CASOS DENTRO DO CODIGO
Em 10 experimentos essa premissa apresentou-se verdadeira, resultando menores correlações, onde nestes casos observou-se a falta dos genes sementes \textbf{\textsl{TP53}} e \textbf{\textsl{AKT1}}. Onde ambos causaram o maior impacto no resultado final ao serem removidos sozinhos do experimento (como foi mencionado na sessão anterior). Estes casos apontam uma correlação do impacto acumulativo da remoção de genes sementes no resultado final.
%%%%%
%
%
%
\subsubsection{Análise dos 100 e 200 primeiros elementos}
%
%Imagem
\begin{figure}[ht!]
\includegraphics[width=1\textwidth]{Images/analyses/fig_S_100_40.pdf}
\includegraphics[width=1\textwidth]{Images/analyses/fig_S_200_40.pdf}
\caption {Análise dos 100 e 200 primeiros elementos ordenados por $\Delta'$.
\label{fig_S_100-200_40}}
\flushleft{Fonte: Produzido pelos autores.}
\end{figure}
%%
%

A figura \ref{fig_S_100-200_40} apresenta dois gráficos comparativos entre agrupamentos de experimentos com variação na quantidade de genes sementes. O gráfico da \textsl{esquerda}, representa a comparação dos \textsl{\textbf{100}} primeiros genes ranqueados em relação ao escore \textsl{\textbf{$\Delta'$}}, onde o eixo \textsl{Horizontal} define os \textsl{\textbf{boxplots}} correspondentes as suas determinadas porcentagens de retirada dos \textsl{\textbf{genes sementes}}. Assim sendo, o eixo \textsl{Vertical} define a porcentagem de interseção dos genes ranqueados pelos experimentos em relação ao experimento original.
Seguindo esta mesma organização, o gráfico da \textsl{direita} representa os \textsl{\textbf{200}} primeiros fenes ranqueados.
%

Podemos notar claramente, que o decrescimento correlacional está presente nos dois gráficos. Ou seja, a correlação dos \textsl{\textbf{genes ranqueados}} cai em mesma proporção nos dois gráficos conforme a quantidade de \textsl{\textbf{genes sementes}} são reduzidas. Porém, podemos enxergar que a \textsl{\textbf{aplitude interqualítica}} respectiva entre os gráficos apresenta uma diminuição. Este aspecto representa bons resultados, pois indica que há uma convergência de resultados conforme o aumento dos \textsl{\textbf{genes ranqueados}} observados.
%

Nesta comparação, podemos observar novamente comportamentos de \textsl{\textbf{outlier}} presentes nos gráficos. Como na análise anterior, os agrupamentos que apresentaram menor correlação, foram os que não tinham em seu agrupamento de \textsl{\textbf{genes sementes}} os mais impactantes observados na etapa de \textsl{\textbf{retirada de uma único gene semente}}, sendo eles \textsl{\textbf{TP53}} e \textsl{\textbf{AKT1}}.
%

Um forte fator de análise é a comparação entre o gráfico dos \textsl{\textbf{10}} primeiros genes ranqueados (\ref{fig_S_10_40}) com o gráfico dos \textsl{\textbf{100}} (\ref{fig_S_100-200_40}). Podemos observar que a mediana subiu de \textsl{\textbf{80\%}} de correlação com a amostra original, para \textsl{\textbf{90\%}} ao comparar os \textsl{\textbf{boxplots}} pertencentes a \textsl{\textbf{10\%}} de remoção. Este fato aponta para uma robustez do método analisado, em vista que fortalece ainda mais o efeito de convergência observado anteriormente. Ao comparar com o gráfico dos \textsl{\textbf{200}} genes ranqueados, notamos que a \textsl{\textbf{mediana}} se mantém a mesma em relação a dos \textsl{\textbf{100}}, indicando que esta convergência ocorre entre nos primeiros \textsl{\textbf{100}} genes ranqueados, sendo este um número muito bom. O mesmo pode ser observado ao comparar os outros \textsl{\textbf{boxplots}} respectivos, os valores apresentados não são os mesmos, mas apresentam um padrão muito próximo de variação.



%
%%%%%
%
%
% ===== X =====
%
%
\subsection{Estudo dos gráficos em relação ao escore \textit{X}}
\subsubsection{Análise dos 10 primeiros elementos}
%
%Imagem
\begin{figure}[ht!]
\centering
\includegraphics[width=\textwidth]{Images/analyses/fig_X_10_40.pdf}
\caption {Análise dos 10 primeiros elementos ordenados por \textit{X}.
\label{fig_X_10_40}}
\flushleft{Fonte: Produzido pelos autores.}
\end{figure}
%
%

%
A Figura~\ref{fig_X_10_40} apresenta um gráfico comparativo dos experimentos utilizando o método \textit{Remoção de mais de um gene}, onde o eixo \textsl{Horizontal} representa a porcentagem de sementes excluídas em relação a amostra original, e o eixo \textsl{Vertical}, por sua vez, representa a porcentagem da interseção dos resultados dos \textsl{\textbf{10}} primeiros genes em relação aos \textsl{\textbf{10}} primeiros apresentados na amostra original, tendo como fator de ranqueamento o escore $X$.
%
Em primeira análise, fica claro que, assim como o ranqueamento pelo escore de ranqueamento $\Delta'$, quanto maior a quantidade de genes sementes removidos nos experimentos, a correlação das listas de priorização gênica diminui. Fator este esperado, em vista que o \textbf{\textsl{Método NERI}} utiliza os genes sementes para realizar a priorização gênica.
%

Ao compararmos este gráfico com o apresentado na análise do escore $\Delta'$ como pode ser vista na Figura~\ref{fig_S_10_40}, podemos notar que a variação dos resultados dos experimentos é muito menor, indicando que ao escore \textbf{$X$} tende a ser mais robusta. O boxplot que intuitivamente apresentaria maior diferença de limite superior e inferior, o referente a \textbf{\textsl{40\%}} de remoção de genes sementes, não apresentou uma grande variação. Este comportamento é o contrário do observado anteriormente, onde a variação anterior apresentou-se em \textbf{\textsl{60\%}}, diferentemente do gráfico em relação ao escore \textbf{$X$} apresentando \textbf{\textsl{30\%}}, metade do valor anterior. Sugerindo mais uma vez a robustez do escore de ranqueamento $X$ superior ao escore $\Delta'$.

Podemos notar que a mediana do pior caso ficou em \textbf{\textsl{60\%}} de similaridade com a lista de genes ranqueados em relação ao experimento original. O pior caso sendo determinado intuitivamente pelo conjunto de experimentos com  \textbf{\textsl{40\%}} de remoção dos genes sementes em relação ao experimento original. A variação entre a menor e maior mediana, sendo elas respectivamente \textbf{\textsl{60\%}} e \textbf{\textsl{90\%}}, apresenta-se em \textbf{\textsl{30\%}}. Um valor muito bom se levarmos em consideração que no pior caso foram removidos \textbf{\textsl{40\%}} dos genes sementes da amostra original, ou seja, a variação do impacto proporcional causado foi menor que o fator de remoção de genes sementes em relação a amostra original. Isto indica uma boa robustez do \textbf{\textsl{Método NERI}} em relação ao fator de ranqueamento $X$.  

\subsubsection{Análise dos 20 e 50 primeiros elementos}
%
%Imagem
\begin{figure}[ht!]
\includegraphics[width=1\textwidth]{Images/analyses/fig_X_20_40.pdf}
\includegraphics[width=1\textwidth]{Images/analyses/fig_X_50_40.pdf}
\caption {Análise dos 20 e 50 primeiros elementos ordenados por \textit{X}.
\label{fig_X_20-50_40}}
\flushleft{Fonte: Produzido pelos autores.}
\end{figure}
%

%
A Figura~\ref{fig_X_20-50_40} apresenta dois gráficos comparativos dos experimentos utilizando o método \textsl{Remoção de mais de um gene}, onde o gráfico da esquerda representa os \textsl{\textbf{20}} primeiros genes ranqueados pelos experimentos e o da direita os primeiros \textsl{\textbf{50}}. Ambos no eixo \textsl{Horizontal} apresentam as porcentagens de genes sementes removidos em relação a amostra original e no eixo \textsl{Vertical}, a porcentagem de similaridade do resultado dos experimentos com o resultado original, ou seja, a similaridade das listas dos experimentos em relação a lista de ranqueamento gênico original.
%

Pode-se notar que, as amplitudes amostrais diminuíram em ambos os casos, isso demonstra uma menor variação dos experimentos em relação a análise feita dos \textsl{\textbf{10}} primeiros genes. Este comportamento, era intuitivamente esperado, em vista que ao aumentar a quantidade de genes selecionados na lista de ranqueamento, a probabilidade da variação dos resultados diminuírem aumenta. Porém, como são muitos genes na rede, o aumento de \textsl{\textbf{10}} e \textsl{\textbf{40}} genes ranqueados em relação a análise anterior, foi o suficiente para identificação. Apesar de ter sido esperado, representa um bom sinal de robustez, de modo que a baixa variação do resultado seja um fator para a mesma.
%

Nota-se também que no gráfico que representa os \textsl{\textbf{20}} primeiros genes ranqueados, quando analisou-se os experimentos que tiveram \textsl{\textbf{40\%}} dos genes sementes removidos em relação a amostra original, apresentaram experimentos \textsl{\textbf{ouliers}} onde alguns representavam uma boa correlação e outros uma má correlação. Isto indica que devemos observar os genes presentes nestes experimentos, para assim tentarmos entender o comportamento diferenciado. Os experimentos \textsl{\textbf{outliers}} que apresentaram uma má correlação, tiveram entre os seus genes sementes removidos os seguintes elementos \textsl{\textbf{TP53}} e \textsl{\textbf{RPGRIP1L}}. Estes genes sementes, são os que apresentaram um maior impacto em sua remoção única na etapa de \textsl{Remoção de um único gene}, onde podemos correlacionar que, o impacto mostra-se acumulativo, ou seja, se um gene com alto impacto em sua remoção é removido juntamente com outro gene causador de um alto impacto, ambos aumentam o impacto total da amostra em questão. Já os experimentos que apresentaram boa correlação, apresentaram a remoção de genes sementes que não causaram grande impacto em sua remoção na etapa de \textsl{Remoção de um único gene}, sendo exemplo destes os elementos \textsl{\textbf{GAD1}} e \textsl{\textbf{HP}}, estes que apresentaram um impacto sempre abaixo de \textsl{\textbf{10\%}}.   
%

Ao observar as medianas dos experimentos, pode-se enxergar que há uma diminuição conforme aumenta a quantidade de genes sementes removidos em relação ao experimento original. Este aspecto indica uma dependência do \textsl{\textbf{Método NERI}} aos genes sementes, fato este já sabido previamente, devido ao mesmo valer-se destes genes para o ranqueamento gênico. O que chama a atenção é o fato da proporção de remoção ser menor que a proporção de impacto causado, ou seja, ao remover \textsl{\textbf{40\%}} dos genes sementes, não impacta o resultado em \textsl{\textbf{40\%}}, mas sim em menos, no caso dos \textsl{\textbf{20}} e \textsl{\textbf{50}} primeiros genes ranqueados, este impacto fica em torno dos \textsl{\textbf{35\%}}.



%
\subsubsection{Análise dos 100 e 200 primeiros elementos}
%
%Imagem
\begin{figure}[ht!]
\includegraphics[width=1\textwidth]{Images/analyses/fig_X_100_40.pdf}
\includegraphics[width=1\textwidth]{Images/analyses/fig_X_200_40.pdf}
\caption {Análise dos 100 e 200 primeiros elementos ordenados por \textit{X}.
\label{fig_X_100-200_40}}
\flushleft{Fonte: Produzido pelos autores.}
\end{figure}
%%
%

%
A Figura~\ref{fig_X_100-200_40} apresenta dois gráficos comparativos em relação aos experimentos que foram gerados através do método \textsl{Remoção de mais de um gene}, onde o gráfico da esquerda representa os \textsl{\textbf{100}} primeiros genes ranqueados pelos experimentos e o da direita os primeiros \textsl{\textbf{200}}. Ambos no eixo \textsl{Horizontal} apresentam as porcentagens de genes sementes removidos em relação a amostra original e no eixo \textsl{Vertical}, a porcentagem de similaridade do resultado dos experimentos com o resultado original, ou seja, a similaridade das listas dos experimentos em relação a lista de ranqueamento gênico original.
%

%
Podemos notar que mesmo os \textsl{outliers} presentes em ambos os gráficos apresentam uma boa correlação com o experimento original. Isto afirma que todos experimentos executados apresentaram um bom resultado de replicabilidade ao analisar os primeiros \textbf{\textsl{100}} e \textbf{\textsl{200}} genes priorizados pelo \textbf{\textsl{Método NERI}}. Este aspecto indica uma forte robustez do método, em vista que mesmo os experimentos que apresentaram comportamento diferente do conjunto no qual estão inseridos obtiveram uma boa correlação com o experimento original.

%
Outro aspecto importante para se observar é o comportamento das correlações dos experimentos, quanto maior a quantidade de genes sementes removidos da amostra original, menor a correlação obtida com o experimento original. Este comportamento esteve presente em todas as análises feitas, tanto no escore $X$ quanto no $\Delta'$, comprovando a dependência do \textbf{\textsl{Método NERI}} em relação aos genes sementes. Porém mesmo assim, apresentou-se robusto a remoção dos genes sementes, indicando bons resultados de replicabilidade. Fato este que torna a utilização do método analisado mais confiável. 
%

%
\subsection{Comparação de \textit{X} em relação a $\Delta'$}
%



TEXTO
%

%
%

\subsection{Observações}
%
\textcolor{red}{=== Texto ===}

%
\newpage
%
\section{Análise geral}
%
\textcolor{red}{======= REESCREVER DAQUI P BAIXO =======}

Podemos considerar que o método tende a ser robusto ao observar os top 10 primeiros genes selecionados, levando em conta seu comportamento em relação a remoção de genes sementes do experimento, porém ainda não podemos tirar conclusões concretas sem antes observar sob outras perspectivas os resultados dos experimentos, como por exemplo, o espalhamento dos resultados em relação ao eixo Y de cada agrupamento, existem amostras que apresentaram resultados muito discrepantes em relação ao conjunto no qual ele pertence.
Para estudar estes casos, alguns aspectos devem ser analisados, sendo um deles, quais genes foram removidos e quais não foram, isso permitirá um melhor entendimento do comportamento dos Outliers.

Neste ponto da pesquisa, surge uma (hipótese, questão, fator) a ser pesquisada nos próximos gráficos a serem estudados, a questão chave neste momento é, se um gene em específico altera demasiadamente o resultado final.
O primeiro passo para responder a este questionamento é o cruzamento dos resultados do Leave One Out, onde será possível mapear o grau de impacto de cada nó exerce sobre o resultado, desta forma verificar entre os genes sementes removidos e 
os titulares destes experimentos outliers e compreender o que aconteceu.


\section{Resultados}
\textcolor{red}{=== TEXTO ===}

\subsection{Dados computacionais}
Os fatores envolvidos no processo de execução dos experimentos, foram as configurações da máquina no qual foi executada e a disponibilidade de tempo de máquina.
As configurações da maquina no qual foram executados os experimentos são as descritas abaixo:
Processador: i7 5 geração
Memória: 16 Gb - 2 pentes 8Gb ddr3 1600Ghz
Armazenamento: 50 Tb HD.
 
Pelo fato de o programa que implementa o método NERI ainda não utilizar paralelismo (utilização de mais de um núcleo de processamento), foram executados 4 instâncias separadas ao mesmo tempo, durante todo a etapa de execução do experimento.

\subsubsection{Consumo de CPU:}
Cada instância ocupou 100\% de processamento de um núcleo físico presente no processador, como o disponível possui 4 núcleos físicos e foram executadas 4 instâncias simultaneamente, o consumo de cpu foi para 100\% do total presente.

\subsubsection{Consumo de Memória:}
Cada instância em execução consumiu em média 1,5 Gb, totalizando por volta de 6Gb alocados por todas as instâncias executantes (4).

\subsubsection{Uso de disco:}
Devido ao fato de os experimentos serem executados em modo Debug, a escrita em arquivo permaneceu constante durante a maior parte do tempo de execução, aumentando somente nos momentos de escrita de resultados.
Os valores consistem em: 78kbs e 1Mbs.

% ----------------------------------------------------------
% Finaliza a parte no bookmark do PDF
% para que se inicie o bookmark na raiz
% e adiciona espaço de parte no Sumário
% ----------------------------------------------------------
\phantompart

% ---
% Conclusão
% ---
\chapter{Conclusão}

Neste trabalho apresentamos a análise de robustez do método NERI, que integra dados biológicos de expressão gênica com dados de redes PPI para priorização de genes relacionados a doenças complexas.
Para explorar a rede PPI, o método parte de alguns nós sementes da rede -- conhecidos por estarem associadas à doença -- e utiliza princípios de importância relativa para explorar a vizinhança das sementes. 
Para explorar a rede PPI o método baseia-se nas hipóteses da \textsl{Network Medicine} combinadas com coexpressão e com isto realiza a priorização gênica através de duas pontuações (escores $X$ e $\Delta'$).
O método NERI obteve bons resultados de replicabilidade em 3 estudos diferentes, mas faltava analisar o quão robusto o método seria caso uma ou algumas sementes fossem removidas.

Para realizar a análise de robustez, alteramos parcialmente o conjunto de 30 genes sementes mas mantivemos os demais dados (ex: expressão KATO inalterados), e em seguida aplicamos o método e comparamos as listas resultantes com as listas originais.
As alterações realizadas no conjunto de gene sementes foram basicamente duas: remoção de um único gene e remoção de vários genes (3, 6, 9 e 12, ou em termos percentuais 10\%, 20\%, 30\%, 40\%) do conjunto de sementes original. 


Em nossos resultados, observamos que a remoção de um único gene semente apresentou maior impacto no score $\Delta'$, ao passo que no escore $X$ apresentou pouco impacto.
Também observamos que os impactos causados em ambos os escores não estão relacionados diretamente ao grau do gene semente removido.
Este é um comportamento importante de citar, pois, a intuição inicial era de que genes com maior grau impactavam mais o resultado em sua remoção. De fato genes com maior grau impactaram no resultado final, mas genes com grau \textsl{1} causaram impacto igual ou muito próximo tornando então inconclusiva a hipótese pré afixada. Desta forma, descobrimos que o grau do gene semente removido não está diretamente relacionado ao impacto causado.


Em relação à remocão de vários genes, considerando o melhor cenário (remoção de \textsl{10\%} das sementes), as listas resultantes do escore $X$ apresentaram em média \textsl{90\%} de interseção com a lista original, e mesmo no pior cenário (remoção de \textsl{40\%} das sementes), a interseção foi de \textsl{60\%} em média, em relação ao score $X$. 
Além disso, observamos  que quanto maior a lista dos primeiros genes comparados, menor é a variância das interseções das listas resultantes com a lista original. Também notamos que os genes sementes com alto grau na rede não influenciam diretamente o resultado final em sua remoção.

Conforme apresentado na Seção~\ref{sec:compXS}, o escore $\Delta'$ sofreu maior impacto que o escore $X$.
Por exemplo, para a comparação dos 10 primeiros genes com remoção de 10\% das sementes, o escore $\Delta'$ atingiu uma interseção mínima de 50\% e mediana de 80\%, ao passo que o score $X$ atingiu mínima de 60\% e mediano de 90\%.
%
Estes valores aumentam ao analisar os resultados com 40\% de remoção dos genes sementes, onde o score $X$ apresenta interseção mínima de 40\% e mediana de 60\%, enquanto o score $\Delta'$ sofreu mais impacto atingindo a interseção mínima de 10\% e mediana de 55\%.

\section{Trabalhos Futuros}

Como trabalhos futuros, indicamos a realização das seguintes tarefas:

\begin{enumerate}

\item Analisar a robustez -- em relação as sementes -- de outros métodos, tais como: \textsl{Random Walk with Restart} e comparar com os resultados obtidos neste trabalho.

\item Pesquisar como integrar novas fontes de dados ao sistema, tais como: dados de epigenética, dados clínicos, etc.
    
\item Concluir interface gráfica adicionando documentação com tutorial de utilização.

\item Disponibilização do código fonte na web, e possivelmente uma publicação em um \textsl{Application Notes}.

\item Criar um serviço web para utilização do método NERI.

\end{enumerate}

% ---



% ----------------------------------------------------------
% ELEMENTOS PÓS-TEXTUAIS
% ----------------------------------------------------------
\postextual
% ----------------------------------------------------------

% ----------------------------------------------------------
% Referências bibliográficas
% ----------------------------------------------------------
\bibliography{abntex2-modelo-references}

% ----------------------------------------------------------
% Glossário
% ----------------------------------------------------------
%
% Consulte o manual da classe abntex2 para orientações sobre o glossário.
%
%\glossary


% ----------------------------------------------------------
% Apêndices
% ----------------------------------------------------------

% ---
% Inicia os apêndices
% ---

\begin{apendicesenv}

%\renewcommand{\appendixtocname}{Apêndices}
%\renewcommand{\appendixpagename}{Apêndices}




% Imprime uma página indicando o início dos apêndices
%\partapendices


\captionsbrazil
\chapter{bbbb}

\chapter{aaaaa}

\chapter{aaaa}


\chapter{Configuração do ambiente}

\noindent \textbf{aaa}

\par aaa





\end{apendicesenv}
% ---
\begin{comment}

% ----------------------------------------------------------
% Anexos
% ----------------------------------------------------------

% ---
% Inicia os anexos
% ---
\begin{anexosenv}

% Imprime uma página indicando o início dos anexos
\partanexos

% ---
\chapter{Morbi ultrices rutrum lorem.}
% ---
\lipsum[30]

% ---
\chapter{Cras non urna sed feugiat cum sociis natoque penatibus et magnis dis
parturient montes nascetur ridiculus mus}
% ---

\lipsum[31]

% ---
\chapter{Fusce facilisis lacinia dui}
% ---

\lipsum[32]

\end{anexosenv}
\end{comment}
%---------------------------------------------------------------------
% INDICE REMISSIVO
%---------------------------------------------------------------------
\phantompart

%---------------------------------------------------------------------

\end{document}