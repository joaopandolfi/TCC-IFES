\chapter[Experimentos, Resultados e Discussão]{Experimentos, Resultados e Discussão}

\section{Análise dos resultados}

	\subsection{Escolha dos métodos de análise dos resultados}
		\subsubsection{Correlação de postos de spearman}
		
		- Calcular correlção de cada seed com a correlação das listas (Leave One Out) - [Impacto do Seed na priozação]
 \-> Tem um cara que calcula medidas de rede (Betwness, Bridgnes e a porra toda)
 \-> Intersecção * Correlação de Spearman

- Definir a medida chamada impacto (O quanto ficou distante o nó), ou seja, o quanto o gene semene fez diferença (Listas priorizadas)
 \-> Checar com todas as medidas de rede (Impacto x Medidas de centralidade)


\section{Resultados}


\subsection{Dados computacionais}
Os fatores envolvidos no processo de execução dos experimentos, foram as configurações da máquina no qual foi executada e a disponibilidade de tempo de máquina.
As configurações da maquina no qual foram executados os experimentos são as descritas abaixo:
Processador: i7 5 geração <Olhar numeração>
Memória: 16 Gb <Olhar marca e velocidade>
Armazenamento: 50 Tb HD.
 
Pelo fato de o programa que implementa o método NERI ainda não utilizar paralelismo (utilização de mais de um núcleo de processamento), foram executados 4 instâncias separadas ao mesmo tempo, durante todo a etapa de execução do experimento.

\subsubsection{Consumo de CPU:}
Cada instância ocupou 100\% de processamento de um núcleo físico presente no processador, como o disponível possui 4 núcleos físicos e foram executadas 4 instâncias simultaneamente, o consumo de cpu foi para 100\% do total presente.

\subsubsection{Consumo de Memória:}
Cada instância em execução consumiu em média 1,5 Gb.

\subsubsection{Uso de disco:}
Devido ao fato de os experimentos serem executados em modo Debug, a escrita em arquivo permaneceu constante durante a maior parte do tempo de execução, aumentando somente nos momentos de escrita de resultados.
Os valores consistem em: 78kbs e 1Mbs.