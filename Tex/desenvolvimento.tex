\chapter[Metodologia]{Metodologia}

Para análise de robustez do método NERI foram utilizados conceitos de validação baseados na alteração dos parâmetros de entrada, de forma que sejam analisados os resultados de saída, para entender o impacto causado devido a estas alterações, no qual consistem na remoção ou não inserção de elementos chaves para o input do programa. 

O processo de validação foi separado em etapas para melhor entendimento e apresentar a temporalidade e cadenciamento dos processos.

\section{Materiais}

Este trabalho utilizou a base de dados <BASE DE DADOS AQUI> que consiste em expressões gênicas de pessoas portadoras da doença <ESQUIZOFRENIA>, estes dados podem ser encontrados <AQUI>. Esta base de dados em específico, foi selecionada esta base devido ao fato de ter sido utilizada na tese de doutorado no qual este trabalho se baseia e se referencia, assim os dados obtidos podem ser comparados com os encontrados e apresentados pelo autor, desta forma evitando o enviesamento do resultado por diferença de experimentação.

O programa NERI também recebe dados de rede de integração proteína proteína (Protein Protein Interaction – PPI), onde os mesmos também foram mantidos os originais utilizados pelo autor, podendo ser encontradas <AQUI>.

Como entrada do sistema também são definidos os Genes Sementes, onde estes são os genes onde há certeza da sua relação com a doença analisada em questão, a base de dados original pode ser encontrada <AQUI>. 


\section{Escolha da variação dos genes sementes}
Neste trabalho, os genes sementes foram escolhidos para variação como parâmetro de entrada, onde o objetivo é identificar o impacto gerado na rede de correlação gênica e no resultado final exibido pelo programa NERI, assim podendo calcular a dependência e sensibilidade do método em relação a qualidade e quantidade de dados dos nós sementes.


\section{Escolha dos métodos de validação}
Os métodos de validação adotados foram selecionados pelas suas caraterísticas de estudo do problema em questão, não podendo deixar margem para enviesamento dos resultados e serem capazes de explorar comportamentos diferentes nos resultados do experimento. Os modelos de validação escolhidos foram: \textit{Leave one Out} e \textit{Cross Validation}.


\subsection{Leave one out}
<VERIFICAR SE FICA  AQUI MESMO>
Leave one out foi aplicado no agrupamento de genes sementes, onde cada execução do programa está faltando um gene semente diferente, de forma que sempre haja a mesma quantidade de genes em cada execução e garantindo que todos tenham ficado de fora pelo menos uma vez, assim sendo, o número final de execuções e amostras de entrada sejam a quantidade total de genes sementes menos 1 (N – 1 | N = total de genes).

Com este método, é possível descobrir, se a falta de um único gene semente é responsável por alterar significantemente o resultado final do método NERI. Desta forma, podendo analisar se o método em questão é sensível a retiradas de nós sementes. 
O Leave one Out também permite a análise de importância relativa dos genes sementes, onde aquele que causar maior impacto no resultado final indica uma importância relativa maior em relação aos outros. 


Porém há outra análise importante que deve ser feita mas o Leave One Out não é capaz de prover, é se a quantidade de nós removidos influenciam diretamente no resultado. Para observar este aspecto, utilizamos o método Cross Validation, no qual, suas características se moldam mais a esta ótica de estudo.


\subsection{Cross Validation}

Este método foi adotado pela sua característica principal, organização de agrupamentos de dados de entrada. Com esta característica chave, buscou-se estudar o comportamento da rede quando há a remoção de mais de um gene semente do agrupamento original de entrada.

Com a formação de agrupamentos de genes sementes aleatórios e de tamanhos variados, pôde-se observar o comportamento do método analisado em situações variadas, buscando o ponto de ruptura de proximidade ao resultado original, assim estimando um grau de dependência e sensibilidade a uma quantidade ou arranjo de genes sementes como parâmetro de entrada.

O fato de os agrupamentos possuírem arranjos de genes diferentes, abre-se a possibilidade de estudo sobre a eficácia de um ou mais genes semente juntos sobre o resultado final, ou seja, se um arranjo específico promove melhores resultados que os demais, sendo estes de mesmo tamanho.

\section{Preparação dos experimentos}

Após determinado os métodos de validação e a base de dados a ser aplicada para análise da ferramenta, o próximo passo é a preparação do experimento a ser desenvolvido, no caso, como a base dados utilizada apresenta 38 genes sementes, a regulação dos parâmetros de remoção para preparação do experimento deve levar em conta diretamente esta quantidade.

Os genes em questão são:
<TABELA DE GENES AQUI>

Estes dados são os brutos de entrada, como o método NERI faz a integração gênica com o GWAS, alguns destes genes, apresentados na tabela acima, não tem representação e ficam de fora, resultando em 30 genes de entrada.

Os genes resultantes são
<TABELA DOS 30 GENES AQUI>

\subsection{Aplicação do método Leave one out}

Para a validação utilizando o método Leave One Out, a preparação do experimento consistiu em gerar entradas para o NERI de forma que cada amostra de entrada tenha um gene a menos do experimento original, onde exista uma entrada para cada gene removido, ou seja, em uma amostra de 38 genes de entrada, temos 37 combinações de entradas possíveis. 
Assim sendo, dado um conjunto de N elementos, a quantidade de entradas possíveis é N-1.

<IMAGEM REPRESENTATIVA>

A aplicação direta no sistema consistem em cada execução independente do programa, o agrupamento de dados de entrada faltar um gene semente diferente, de forma que sempre haja a mesma quantidade de genes em cada execução e garantindo que todos os genes tenham ficado de fora pelo menos uma vez no total de execuções independentes.

<Para preparação destas entradas, foi desenvolvido um script em Python 3.X para automatização do processo e para evitar falha humana.>

<Script Python GERA_LOO>

Para ilustração do experimento, segue o exemplo abaixo:
Temos a amostra original A sendo: 1,2,3,4,5
Os subconjuntos gerados utilizando o conceito de Leave One Out são:
As1: 2,3,4,5
As2: 1,3,4,5
As3: 1,2,4,5
As4: 1,2,3,4

<IMAGEM ToyExampleLOO1>

\subsection{Aplicação do método Cross Validation}

Para melhor aproveitamento do método escolhido, o primeiro passo a ser dado, é a definição de tamanho dos agrupamentos de dados de entrada para cada bateria de execuções.
Levando em consideração a quantidade de genes de entrada total do experimento, 30 após a integração com o GWAS, definiu-se que as remoções seriam feitas em relação a porcentagem da amostra original, sendo as porcentagens definidas 10\% (3 genes), 20\% (6 genes), 30\% (9 genes) e 40\% (12 genes).

Com as porcentagens de remoção definidos, determinou-se a quantidade de agrupamentos de dados de entrada a serem executados para cada etapa, onde estão descritos na tabela abaixo

<COLOCAR EM TABELA>
10\% --- 50 execuções
20\% --- 50 execuções
30\% --- 50 execuções
40\% --- 50 execuções

Cada agrupamento deve ser diferente do outro, de forma que o conjunto de genes removidos não se repita dentro de cada etapa, em vista que, caso isso aconteça, a análise final será comprometida por possuir resultados iguais provenientes de entradas de dados iguais.

<Para garantir a diferença entre os dados de entrada, foi desenvolvido um script em Python 3.x para automatização da tarefa evitando falha humana no processo.>
<Script GERA_CVV>

Para ilustração do experimento, segue o exemplo abaixo:

Temos a amostra original A, sendo: 0,1,2,3,4,5,6,7,8,9.
Determinado o fator de remoção em 20%, e a quantidade de dados de entrada em 5.
Dado 20% de 10 elementos, temos fator de remoção = 2 elementos.
Temos os subconjuntos:
As1 = 0,1,2,3,4,5,6,7
As2 = 0,1,2,3,4,5,6,8
As3 = 0,1,2,3,4,5,6,9
As4 = 1,2,3,4,5,6,7,8
As5 = 2,3,4,5,6,7,8,9

Sendo os removidos
Rem As1 = 8,9
Rem As2 = 7,9
Rem As3 = 7,8
Rem As4 = 0,9
Rem As5 = 0,1 

<IMG_TOY_EXAMPLE_CV>

Após os agrupamentos de dados de entrada serem preparados, o programa principal é executado individualmente para cada agrupamento. Totalizam-se 200 execuções individuais para a aplicação da validação cruzada.


\section{Execução dos experimentos}

Nesta etapa, os experimentos encontram-se preparados para execução direta no programa principal que implementa o método NERI, a somatória de execuções totais a serem efetuadas com as técnicas de validação escolhidas consiste em 230 chamadas separadas.
Devido ao fato de o programa realizar cálculos demorados, houve a necessidade de automatização do processo de execução, onde foi desenvolvido um script em Shell (linha de comando Linux), para efetuação do trabalho. <REF AO APENDICE>
Um outro quesito no qual influenciou diretamente na execução dos experimentos, foi o fato de o programa original não possuir um esquema de diretórios robusto e chamadas pelo terminal preparada para este tipo de utilização em massa, acarretou na alteração estrutural do programa original para organização dos dados de entrada e dos resultados apresentados, onde também foi desenvolvida uma interface gráfica e uma interface em linha de comando (CLI), para facilitar a utilização por pessoas que não tem familiaridade com este tipo de utilização de programas.



%Exemplo codigo no TEX
%\begin{lstlisting}[caption={Código fonte do método getBlockStatic da classe %Tools.},label=getBlockStatic,language=Java]
%public static List<Rect> getBlockStatic(Bitmap bmp, int nr, int nrTotal) {
%    Mat mat = bitmapToMat(bmp);
%    Size size = mat.size();
%    List<Rect> list = new ArrayList<>();
%    int left, top, rigth, bottom;
%    for (int i = 0; i < nrTotal; i++) {
%        top = 0;
%        if (i == 0) {
%            left = 1;
%        } else {
%            left = 1 + (mat.cols() / nrTotal * i);
%        }
%        bottom = mat.rows();
%        rigth = left + (mat.cols() / nrTotal);
%        list.add(new Rect(left, top, rigth - 1, bottom - 1));
%    }
%    List<Rect> result = new ArrayList<>();
%    for (int j = nr - 1; j < 4; j++) {
%        result.add(list.get(j));
%    }
%    list = null;
%    return result;
%}
%\end{lstlisting}