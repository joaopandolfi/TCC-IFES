\begin{resumo}

Um dos grandes problemas enfrentados pelos pesquisadores é o estudo das doenças complexas, pois elas são poligênicas e multifatoriais, fazendo com que diferentes estudos apresentem baixa replicabilidade.
Recentemente, avanços significativos tem sido obtidos por métodos que realizam integração de dados entre expressão gênica e dados de rede PPI (\textit{Protein Protein Interaction Network}).
Dentre eles destaca-se o método NERI que obteve bons resultados de replicabilidade.
Esse método baseia-se nas hipóteses da \textit{Network Medicine} combinadas com métodos de importância relativa em redes complexas.
A importância relativa é uma forma de inferir a relevância topológica dos nós da rede baseado em um conjunto de nós conhecidos como sementes. Entretanto, esse método carece de uma análise de robustez, que avalie o quanto seus resultados são dependentes dos genes sementes.
Neste trabalho, analisamos a robustez do método NERI com relação aos genes sementes visando avaliar o impacto da remoção progressiva destes.
Realizamos experimentos mantendo fixos a rede PPI e os dados de expressão, 
mas removendo progressivamente parte dos nós sementes do conjunto original (de forma similar as técnicas de avaliação de classificadores \textit{leave-one-out} e validação cruzada), e comparando a interseção entre seus resultados com o resultado original.
%
Variamos o percentual de genes sementes excluídos entre 10 e 40\% e, em seguida, comparamos os primeiros elementos das listas resultantes com os primeiros da lista original.
%
Considerando o melhor cenário (remoção de 10\% das sementes), as listas resultantes apresentaram em média 90\% de interseção com a lista original, e mesmo no pior cenário (remoção de 40\% das sementes), a interseção foi de 60\% em média.
Além disso, observamos também que quanto maior a lista dos primeiros genes comparados, menor é a variância das interseções das listas resultantes com a lista original.
%
Portanto, o método NERI pode ser considerado robusto com relação aos genes sementes, indica replicabilidade mesmo em situações onde os genes sementes são variados.


Palavras chaves: Network Medicine, Validação Cruzada, Leave-one-out, Robustez.
\end{resumo}

An importante problem faced by researchers is the study of complex diseases, because they are polygenic and multifactorial, implying small replicability among different studies.
Recently, significant advances have been reached by methods that perform data integration between gene expression and PPI (Protein-Protein Interaction Network) data.
Among them, we highlight the NERI method which achieved good replicability results.
This method is based on Network Medicine hypotheses combined with relative importance analyses in complex networks.
Relative importance assesses the topological relevance of network nodes based on a set of important nodes known as seeds.
However, until this date no robustness analysis was conducted for the NERI method, in order to evaluate how much its results are dependent on the seed genes.
In this work, we analyzed the robustness of the NERI method with regard to the seed genes in order to evaluate the impact of their progressive removal.
We performed experiments fixing the PPI network and expression data, but progressively removing parts of the seed nodes from the original set (analogous to the classification assessment techniques, such as cross-validation), and comparing the intersection between their results and the original result.
We excluded between 10%--40% of the seed genes and compared the top elements of the resulting lists with the top ones from the original list.
Considering the best scenario (removal of 10% of seeds), the resulting lists averaged 90% of intersection with the original list, and even in the worst case scenario (removal of 40% of seeds), the intersection was 60% in average.
In addition, we also note that the larger the list of the first genes compared, the smaller is the variance of the intersections of the resulting lists with the original list.
Therefore, the NERI method can be considered robust with respect to the exclusion of seed genes, also presenting good replicability in such a case.