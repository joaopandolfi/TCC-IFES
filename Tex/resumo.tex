\begin{resumo}

Um dos grandes problemas enfrentados pelos pesquisadores é o estudo das doenças complexas, pois elas são poligênicas e multifatoriais, fazendo com que diferentes estudos apresentem baixa replicabilidade.
Recentemente, avanços significativos tem sido obtidos por métodos que realizam integração de dados entre expressão gênica e dados de rede PPI (\textit{Protein Protein Interaction Network}).
Dentre eles destaca-se o método NERI que obteve bons resultados de replicabilidade.
Esse método baseia-se nas hipóteses da \textit{Network Medicine} combinadas com métodos de importância relativa em redes complexas.
A importância relativa é uma forma de inferir a relevância topológica dos nós da rede baseado em um conjunto de nós conhecidos como sementes. Entretanto, esse método carece de uma análise de robustez, que avalie o quanto seus resultados são dependentes dos genes sementes.
Neste trabalho, analisamos a robustez do método NERI com relação aos genes sementes visando avaliar o impacto da remoção progressiva destes.
Realizamos experimentos mantendo fixos a rede PPI e os dados de expressão, 
mas removendo progressivamente parte dos nós sementes do conjunto original (de forma similar as técnicas de avaliação de classificadores \textit{leave-one-out} e validação cruzada), e comparando a interseção entre seus resultados com o resultado original.
%
Variamos o percentual de genes sementes excluídos entre 10 e 40\% e, em seguida, comparamos os primeiros elementos das listas resultantes com os primeiros da lista original.
%
Considerando o melhor cenário (remoção de 10\% das sementes), as listas resultantes apresentaram em média 90\% de interseção com a lista original, e mesmo no pior cenário (remoção de 40\% das sementes), a interseção foi de 60\% em média.
Além disso, observamos também que quanto maior a lista dos primeiros genes comparados, menor é a variância das interseções das listas resultantes com a lista original.
%
Portanto, o método NERI pode ser considerado robusto com relação aos genes sementes, o que indica replicabilidade mesmo em situações onde os genes sementes são variados.


Palavras chaves: Network Medicine; Rede PPI; Importância Relativa; Método NERI; Robustez das Sementes.
\end{resumo}
