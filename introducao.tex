%tabulacao para a listagem
\newcommand{\itab}[1]{\hspace{0em}\rlap{#1}}
\newcommand{\tab}[1]{\hspace{.2\textwidth}\rlap{#1}}
%inicio do capitulo
\chapter[Introdução]{Introdução}

%\textit{} // Coloca em italico
%\cite{} //Cita autor
%\ref{} //Cita figura

Doenças complexas são poligênicas e multifatoriais, ou seja, além de serem causadas por mutações em mais de um gene, também são influenciadas por fatores ambientais \cite{davey-mith}. Como título de informação, alguns exemplos de doenças complexas são doença de Parkinson e esclerose múltipla \cite{Hunter-2005}. Quanto aos fatores genéticos, devido ao fato destas doenças serem poligênicas, as mutações podem levar a uma propagação não natural de informação e sinais, de forma que afete outros genes e/ou mecanismos dependentes dos que sofreram determinada mutação. 


% \par
% \begin{figure}[ht!]
% \centering
% \includegraphics[width=150mm]{Images/pesquisaCrescimentoNumSmarth.png}
% \caption{Compras feitas através de dispositivos móveis. Imagem retirada de \cite{MarketingCharts2014}} \label{imagemDigitalCap1}
% \end{figure}


Uma forma de estudar este tipo de doença, é analisar os transcritos gerados pela transcrição dos genes, de forma a buscar uma relação de co-expressão, tendo como objetivo encontrar genes que influenciam na doença em questão. Uma forma de utilizar estes dados, é fazer a modelagem em forma de rede, onde cada nó representa um gene, as arestas representam a co-expressão genica, e ao utilizar a topologia de redes com pesos, o fator de co-expressão torna-se então o peso, determinando assim o grau de relacionamento entre dois nós, com essa abordagem, é possível aplicar conceitos e propriedades de grafos no problema, devido ao fato de ele estar modelado em rede.
Outra forma de estudar as doenças poligênicas, é analisar as interações entre proteínas (\textit{PPI – Protein-Protein Interaction}), onde também é aplicada a abordagem de redes para investigação da doença, no qual é chamada de hipótese da \textit{Network Medicine} \cite{barabasi}. Este modelo leva em conta o nível de interação entre as proteínas e quais foram os genes responsáveis por gerá-las, podendo assim ter um mapeamento gênico e proteico ao mesmo tempo.

Estas duas abordagens citadas englobam conceitos de redes complexas, onde têm-se a representação de dados e relações entre eles em forma de grafos, sejam eles com pesos ou não (em sua grande maioria são utilizados grafos direcionados e com peso), esta abordagem permite utilizar conceitos fundamentados sobre teoria de grafos e algoritmos consolidados para análise do problema, ganhando-se assim mais ferramentas para tratamento do modelo em questão.
Como por exemplo, algoritmos de caminho mínimo, onde visam encontrar o menor caminho entre dois nós, na genética, cada aresta é uma relação entre os genes (\textit{nós}), portanto, quanto o menor caminho entre dois genes (\textit{nós}), mais próxima é a sua relação. 

De acordo com os conceitos apresentados, existem diversas abordagens para tratar doenças poligênicas, dentre elas, destaca-se o método NERI que apresentou bons resultados de replicabilidade. Este é um método que baseia-se em importância relativa, ou seja, fundamenta-se em nós sementes para o seu funcionamento, onde estes são genes sabidamente reconhecidos como importantes. Em vista desta abordagem, este método carece de uma análise de robustez, o que significa analisar o quão dependente dos nós sementes o método é, de forma a encontrar um coeficiente de confiança, para assim gerar uma segurança na utilização da ferramenta, porém, para encontrar estes coeficientes é necessário observar o comportamento do método quando há retiradas de nós sementes da rede de entrada.

Ao analisar o código fonte do programa em questão, identificamos a necessidade de reestruturação do mesmo, de forma que fique mais modularizado, facilitando a manutenção e adição de novas funcionalidades futuras, também temos como objetivo facilitar e incentivar a colaboração de pesquisadores no desenvolvimento futuro da ferramenta, pelo fato de o código ser aberto, ou seja, qualquer um que estiver disposto a contribuir terá acesso ao código fonte.
Como a ferramenta foi desenvolvida para ser utilizada por biólogos, identificamos também, a necessidade de desenvolver uma interface gráfica para facilitar e disseminar o uso. Atualmente a utilização é feita somente por linha de comando no terminal, o que requer um nível de conhecimento mínimo. A interface gráfica tem como objetivo viabilizar o uso do programa para biólogos não familiarizados em utilizar programas por linhas de comando em terminal, potencializando assim o alcance da ferramenta e incentivando o uso de métodos computacionais por biólogos.


%Exemplo de imagem

%\textbf{\begin{figure}[ht!]
%\centering
%\includegraphics[width=130mm]{Images/debit.jpg}
%\caption {A evolução das máquinas de cartão. \label{adas}}
%\flushleft{Fonte: Imagem de retirada e adaptada de \cite{evolu}.}
%\end{figure}}

%\begin{figure}[ht!]
%\centering
%\includegraphics[width=130mm]{Images/debit.jpg}
%\caption {A evolução das máquinas de cartão. \label{Evolucao}}
%\flushleft{Fonte: Imagem de retirada e adaptada de \cite{evolu}.}
%\end{figure}


%\section{OCR} // Criando sessao



\section{Objetivos}
\subsection{Objetivo Geral}
\begin{center}
  \begin{enumerate}
  \item {Analisar robustez do método NERI para avaliar o impacto da retirada de alguns genes sementes.}

  \end{enumerate}
\end{center}

\subsection{Objetivos Específicos}
\begin{center}
  \begin{enumerate}
  \item {Refatorar o código para facilitar a manutenção, utilização e contribuição externa da comunidade de usuários e desenvolvedores.}
    \item {Implementar os algoritmos de validação cruzada e \textit{leave-one-out} aplicados ao método NERI.}
    \item{Implementar interface gráfica para facilitar o uso da ferramenta, visando torná-la mais intuitiva e amigável aos pesquisadores da área biológica.}
  \item{Desenvolver um módulo de \textit{Template Matching}.}
    \item {Analisar a robustez do método NERI verificando sua dependência em relação aos nós sementes.}
  \end{enumerate}
\end{center}


\section{Organização do trabalho}
Escrever

